%% abtex2-modelo-trabalho-academico.tex, v-1.9.7 laurocesar
%% Copyright 2012-2018 by abnTeX2 group at http://www.abntex.net.br/ 
%%
%% This work may be distributed and/or modified under the
%% conditions of the LaTeX Project Public License, either version 1.3
%% of this license or (at your option) any later version.
%% The latest version of this license is in
%%   http://www.latex-project.org/lppl.txt
%% and version 1.3 or later is part of all distributions of LaTeX
%% version 2005/12/01 or later.
%%
%% This work has the LPPL maintenance status `maintained'.
%% 
%% The Current Maintainer of this work is the abnTeX2 team, led
%% by Lauro C?sar Araujo. Further information are available on 
%% http://www.abntex.net.br/
%%
%% This work consists of the files abntex2-modelo-trabalho-academico.tex,
%% abntex2-modelo-include-comandos and abntex2-modelo-references.bib
%%

% ------------------------------------------------------------------------
% ------------------------------------------------------------------------
% abnTeX2: Modelo de Trabalho Academico (tese de doutorado, dissertacao de
% mestrado e trabalhos monograficos em geral) em conformidade com 
% ABNT NBR 14724:2011: Informacao e documentacao - Trabalhos academicos -
% Apresentacao
% ------------------------------------------------------------------------
% ------------------------------------------------------------------------

\documentclass[
	% -- op??es da classe memoir --
	12pt,				% tamanho da fonte
	openright,			% cap?tulos come?am em p?g ?mpar (insere p?gina vazia caso preciso)
	twoside,			% para impress?o em recto e verso. Oposto a oneside
	a4paper,			% tamanho do papel. 
	% -- op??es da classe abntex2 --
	%chapter=TITLE,		% t?tulos de cap?tulos convertidos em letras mai?sculas
	%section=TITLE,		% t?tulos de se??es convertidos em letras mai?sculas
	%subsection=TITLE,	% t?tulos de subse??es convertidos em letras mai?sculas
	%subsubsection=TITLE,% t?tulos de subsubse??es convertidos em letras mai?sculas
	% -- op??es do pacote babel --
	english,			% idioma adicional para hifeniza??o
	french,				% idioma adicional para hifeniza??o
	spanish,			% idioma adicional para hifeniza??o
	brazil				% o ?ltimo idioma ? o principal do documento
	]{abntex2}

% ---
% Pacotes b?sicos 
% ---
\usepackage{lmodern}			% Usa a fonte Latin Modern			
\usepackage[T1]{fontenc}		% Selecao de codigos de fonte.
\usepackage[utf8]{inputenc}		% Codificacao do documento (convers?o autom?tica dos acentos)
\usepackage{indentfirst}		% Indenta o primeiro par?grafo de cada se??o.
\usepackage{color}				% Controle das cores
\usepackage{graphicx}			% Inclus?o de gr?ficos
\usepackage{microtype} 			% para melhorias de justifica??o
% ---
		
% ---
% Pacotes adicionais, usados apenas no ?mbito do Modelo Can?nico do abnteX2
% ---
%\usepackage{lipsum}				% para gera??o de dummy text
% ---

% ---
% Pacotes de cita??es
% ---
\usepackage[brazilian,hyperpageref]{backref}	 % Paginas com as cita??es na bibl
%\usepackage[alf]{abntex2cite}	% Cita??es padr?o ABNT

% --- 
% CONFIGURA??ES DE PACOTES
% --- 

% ---
% Configura??es do pacote backref
% Usado sem a op??o hyperpageref de backref
\renewcommand{\backrefpagesname}{Citado na(s) p?gina(s):~}
% Texto padr?o antes do n?mero das p?ginas
\renewcommand{\backref}{}
% Define os textos da cita??o
\renewcommand*{\backrefalt}[4]{
	\ifcase #1 %
		Nenhuma cita??o no texto.%
	\or
		Citado na p?gina #2.%
	\else
		Citado #1 vezes nas p?ginas #2.%
	\fi}%
% ---

% ---
% Capa modificada
%---
\renewcommand{\imprimircapa}{%
  \begin{capa}%
    \center
    \ABNTEXchapterfont\Large UNIVERSIDADE QUE TORNA OS DOCUMENTOS AINDAMAIS ESTRANHOS
    \vspace*{1cm}
    
    {\ABNTEXchapterfont\large\imprimirautor}
    
    \vfill
    \begin{center}
    \ABNTEXchapterfont\bfseries\LARGE\imprimirtitulo
    \end{center}
    \vfill
    
    \large\imprimirlocal
    \large\imprimirdata
    
    \vspace*{1cm}
  \end{capa}}
}
% ---
% Informa??es de dados para CAPA e FOLHA DE ROSTO
% ---
\titulo{$title$}
\autor{$author$}
\data{$date$}
\orientador{$orientador$}
$if(coorientadors)$
\coorientador{$coorientador$}
$endif$
\instituicao{%
  UNIVERSIDADE FEDERAL DO RIO GRANDE DO SUL
  \par
  INSTITUTO DE BIOCI?NCIAS
  \par
  PROGRAMA DE P?S-GRADUA??O EM ECOLOGIA}
\tipotrabalho{$tipotrabalho$}
% O preambulo deve conter o tipo do trabalho, o objetivo, 
% o nome da institui??o e a ?rea de concentra??o 
\preambulo{$preambulo$}
% ---


% ---
% Configura??es de apar?ncia do PDF final

% alterando o aspecto da cor azul
\definecolor{blue}{RGB}{41,5,195}

% informa??es do PDF
\makeatletter
\hypersetup{
     	%pagebackref=true,
		pdftitle={\@title}, 
		pdfauthor={\@author},
    	pdfsubject={\imprimirpreambulo},
	    pdfcreator={LaTeX with abnTeX2},
		pdfkeywords={abnt}{latex}{abntex}{abntex2}{trabalho acad?mico}, 
		colorlinks=true,       		% false: boxed links; true: colored links
    	linkcolor=blue,          	% color of internal links
    	citecolor=blue,        		% color of links to bibliography
    	filecolor=magenta,      		% color of file links
		urlcolor=blue,
		bookmarksdepth=4
}
\makeatother
% --- 

% ---
% Posiciona figuras e tabelas no topo da p?gina quando adicionadas sozinhas
% em um p?gina em branco. Ver https://github.com/abntex/abntex2/issues/170
\makeatletter
\setlength{\@fptop}{5pt} % Set distance from top of page to first float
\makeatother
% ---

% ---
% Possibilita cria??o de Quadros e Lista de quadros.
% Ver https://github.com/abntex/abntex2/issues/176
%
% \newcommand{\quadroname}{Quadro}
% \newcommand{\listofquadrosname}{Lista de quadros}
% 
% \newfloat[chapter]{quadro}{loq}{\quadroname}
% \newlistof{listofquadros}{loq}{\listofquadrosname}
% \newlistentry{quadro}{loq}{0}
% 
% % configura??es para atender ?s regras da ABNT
% \setfloatadjustment{quadro}{\centering}
% \counterwithout{quadro}{chapter}
% \renewcommand{\cftquadroname}{\quadroname\space} 
% \renewcommand*{\cftquadroaftersnum}{\hfill--\hfill}
% 
% \setfloatlocations{quadro}{hbtp} % Ver https://github.com/abntex/abntex2/issues/176
% % ---

% --- 
% Espa?amentos entre linhas e par?grafos 
% --- 

% O tamanho do par?grafo ? dado por:
\setlength{\parindent}{1.3cm}

% Controle do espa?amento entre um par?grafo e outro:
\setlength{\parskip}{0.2cm}  % tente tamb?m \onelineskip

%Line spacing
\renewcommand{\baselinestretch}{1.6} %Value 	Line spacing: 1.0 	single spacing, 1.3 	one-and-a-half spacing, 1.6 	double spacing.

% ---
% compila o indice
% ---
%\makeindex
% ---

% ----
% In?cio do documento
% ----
\begin{document}

% Seleciona o idioma do documento (conforme pacotes do babel)
%\selectlanguage{english}
\selectlanguage{brazil}

% Retira espa?o extra obsoleto entre as frases.
\frenchspacing 

% ----------------------------------------------------------
% ELEMENTOS PR?-TEXTUAIS
% ----------------------------------------------------------
% \pretextual

% ---
% Capa
% ---
\imprimircapa
% ---

% ---
% Folha de rosto
% (o * indica que haver? a ficha bibliogr?fica)
% ---
\imprimirfolhaderosto*
% ---

% ---
% Inserir a ficha bibliografica
% ---

% Isto ? um exemplo de Ficha Catalogr?fica, ou ``Dados internacionais de
% cataloga??o-na-publica??o''. Voc? pode utilizar este modelo como refer?ncia. 
% Por?m, provavelmente a biblioteca da sua universidade lhe fornecer? um PDF
% com a ficha catalogr?fica definitiva ap?s a defesa do trabalho. Quando estiver
% com o documento, salve-o como PDF no diret?rio do seu projeto e substitua todo
% o conte?do de implementa??o deste arquivo pelo comando abaixo:
%
% \begin{fichacatalografica}
%     \includepdf{fig_ficha_catalografica.pdf}
% \end{fichacatalografica}

\begin{fichacatalografica}
	\sffamily
	\vspace*{\fill}					% Posi??o vertical
	\begin{center}					% Minipage Centralizado
	\fbox{\begin{minipage}[c][8cm]{13.5cm}		% Largura
	\small
	\imprimirautor
	%Sobrenome, Nome do autor
	
	\hspace{0.5cm} \imprimirtitulo  / \imprimirautor. --
	\imprimirlocal, \imprimirdata-
	
	\hspace{0.5cm} \thelastpage p. : il. (algumas color.) ; 30 cm.\\
	
	\hspace{0.5cm} \imprimirorientadorRotulo~\imprimirorientador\\
	
	\hspace{0.5cm}
	\parbox[t]{\textwidth}{\imprimirtipotrabalho~--~\imprimirinstituicao,
	\imprimirdata.}\\
	
	\hspace{0.5cm}
		1. Palavra-chave1.
		2. Palavra-chave2.
		2. Palavra-chave3.
		I. Orientador.
		II. Universidade xxx.
		III. Faculdade de xxx.
		IV. T?tulo 			
	\end{minipage}}
	\end{center}
\end{fichacatalografica}
% ---

% ---
% % Inserir errata
% % ---
% \begin{errata}
% Elemento opcional da \citeonline[4.2.1.2]{NBR14724:2011}. Exemplo:
% 
% \vspace{\onelineskip}
% 
% FERRIGNO, C. R. A. \textbf{Tratamento de neoplasias ?sseas apendiculares com
% reimplanta??o de enxerto ?sseo aut?logo autoclavado associado ao plasma
% rico em plaquetas}: estudo cr?tico na cirurgia de preserva??o de membro em
% c?es. 2011. 128 f. Tese (Livre-Doc?ncia) - Faculdade de Medicina Veterin?ria e
% Zootecnia, Universidade de S?o Paulo, S?o Paulo, 2011.
% 
% \begin{table}[htb]
% \center
% \footnotesize
% \begin{tabular}{|p{1.4cm}|p{1cm}|p{3cm}|p{3cm}|}
%   \hline
%    \textbf{Folha} & \textbf{Linha}  & \textbf{Onde se l?}  & \textbf{Leia-se}  \\
%     \hline
%     1 & 10 & auto-conclavo & autoconclavo\\
%    \hline
% \end{tabular}
% \end{table}
% 
% \end{errata}
% % ---
% 
% % ---
% % Inserir folha de aprova??o
% % ---
% 
% % Isto ? um exemplo de Folha de aprova??o, elemento obrigat?rio da NBR
% % 14724/2011 (se??o 4.2.1.3). Voc? pode utilizar este modelo at? a aprova??o
% % do trabalho. Ap?s isso, substitua todo o conte?do deste arquivo por uma
% % imagem da p?gina assinada pela banca com o comando abaixo:
% %
% % \begin{folhadeaprovacao}
% % \includepdf{folhadeaprovacao_final.pdf}
% % \end{folhadeaprovacao}
% %
% \begin{folhadeaprovacao}
% 
%   \begin{center}
%     {\ABNTEXchapterfont\large\imprimirautor}
% 
%     \vspace*{\fill}\vspace*{\fill}
%     \begin{center}
%       \ABNTEXchapterfont\bfseries\Large\imprimirtitulo
%     \end{center}
%     \vspace*{\fill}
%     
%     \hspace{.45\textwidth}
%     \begin{minipage}{.5\textwidth}
%         \imprimirpreambulo
%     \end{minipage}%
%     \vspace*{\fill}
%    \end{center}
%         
%    Trabalho aprovado. \imprimirlocal, 24 de novembro de 2012:
% 
%    \assinatura{\textbf{\imprimirorientador} \\ Orientador} 
%    \assinatura{\textbf{Professor} \\ Convidado 1}
%    \assinatura{\textbf{Professor} \\ Convidado 2}
%    %\assinatura{\textbf{Professor} \\ Convidado 3}
%    %\assinatura{\textbf{Professor} \\ Convidado 4}
%       
%    \begin{center}
%     \vspace*{0.5cm}
%     {\large\imprimirlocal}
%     \par
%     {\large\imprimirdata}
%     \vspace*{1cm}
%   \end{center}
%   
% \end{folhadeaprovacao}
% % ---
% 
% % ---
% % Dedicat?ria
% % ---
% \begin{dedicatoria}
%    \vspace*{\fill}
%    \centering
%    \noindent
%    \textit{ Este trabalho ? dedicado ?s crian?as adultas que,\\
%    quando pequenas, sonharam em se tornar cientistas.} \vspace*{\fill}
% \end{dedicatoria}
% % ---

% ---
% Agradecimentos
% ---
\begin{agradecimentos}
$agradecimentos$
\end{agradecimentos}
% ---

% ---
% Ep?grafe
% ---
\begin{epigrafe}
    \vspace*{\fill}
	\begin{flushright}
		\textit{$epigrafe$}
	\end{flushright}
\end{epigrafe}
% ---

% ---
% RESUMOS
% ---

% resumo em portugu?s
\setlength{\absparsep}{18pt} % ajusta o espa?amento dos par?grafos do resumo
\begin{resumo}
 $resumo$
 \textbf{Palavras-chave}: $palavraschave$
\end{resumo}

% resumo em ingl?s
\begin{resumo}[Abstract]
 \begin{otherlanguage*}{english}
   $abstract$

   \vspace{\onelineskip}
 
   \noindent 
   \textbf{Keywords}: $keywords$
 \end{otherlanguage*}
\end{resumo}

% % resumo em franc?s 
% \begin{resumo}[R?sum?]
%  \begin{otherlanguage*}{french}
%     Il s'agit d'un r?sum? en fran?ais.
%  
%    \textbf{Mots-cl?s}: latex. abntex. publication de textes.
%  \end{otherlanguage*}
% \end{resumo}
% 
% % resumo em espanhol
% \begin{resumo}[Resumen]
%  \begin{otherlanguage*}{spanish}
%    Este es el resumen en espa?ol.
%   
%    \textbf{Palabras clave}: latex. abntex. publicaci?n de textos.
%  \end{otherlanguage*}
% \end{resumo}
% % ---

% ---
% inserir o sumario
% ---
\pdfbookmark[0]{\contentsname}{toc}
\tableofcontents*
\cleardoublepage
% ---

% % ---
% % inserir lista de quadros
% % ---
% \pdfbookmark[0]{\listofquadrosname}{loq}
% \listofquadros*
% \cleardoublepage
% % ---

% ---
% inserir lista de ilustra??es
% ---
\pdfbookmark[0]{\listfigurename}{lof}
\listoffigures*
\cleardoublepage
% ---

% ---
% inserir lista de tabelas
% ---
\pdfbookmark[0]{\listtablename}{lot}
\listoftables*
\cleardoublepage
% ---

% ---
% inserir lista de abreviaturas e siglas
% ---
\begin{siglas}
  \item[ABNT] Associa??o Brasileira de Normas T?cnicas
  \item[abnTeX] ABsurdas Normas para TeX
\end{siglas}

% ----------------------------------------------------------
% ELEMENTOS TEXTUAIS
% ----------------------------------------------------------
\textual

% all your chapters and appendices will appear here
$body$

\postextual
% ----------------------------------------------------------

% ----------------------------------------------------------
% Refer?ncias bibliogr?ficas
% ----------------------------------------------------------
\bibliography{abntex2-modelo-references}

% ----------------------------------------------------------
% Gloss?rio
% ----------------------------------------------------------
%
% Consulte o manual da classe abntex2 para orienta??es sobre o gloss?rio.
%
%\glossary

% ----------------------------------------------------------
% Ap?ndices
% ----------------------------------------------------------

% ---
% Inicia os ap?ndices
% ---
\begin{apendicesenv}

% Imprime uma p?gina indicando o in?cio dos ap?ndices
\partapendices
$apendices$
\end{apendicesenv}
% ---

% 
% % ----------------------------------------------------------
% % Anexos
% % ----------------------------------------------------------
% 
% % ---
% % Inicia os anexos
% % ---
% \begin{anexosenv}
% 
% % Imprime uma p?gina indicando o in?cio dos anexos
% \partanexos
% 
% % ---
% \chapter{Morbi ultrices rutrum lorem.}
% % ---
% \lipsum[30]
% 
% % ---
% \chapter{Cras non urna sed feugiat cum sociis natoque penatibus et magnis dis
% parturient montes nascetur ridiculus mus}
% % ---
% 
% \lipsum[31]
% 
% % ---
% \chapter{Fusce facilisis lacinia dui}
% % ---
% 
% \lipsum[32]
% 
% \end{anexosenv}
% 
% %---------------------------------------------------------------------
% % INDICE REMISSIVO
% %---------------------------------------------------------------------
% \phantompart
% \printindex
% %---------------------------------------------------------------------

\end{document}