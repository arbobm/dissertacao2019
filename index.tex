\documentclass[12pt,]{article}
\usepackage{lmodern}
\usepackage{amssymb,amsmath}
\usepackage{ifxetex,ifluatex}
\usepackage{fixltx2e} % provides \textsubscript
\ifnum 0\ifxetex 1\fi\ifluatex 1\fi=0 % if pdftex
  \usepackage[T1]{fontenc}
  \usepackage[utf8]{inputenc}
\else % if luatex or xelatex
  \ifxetex
    \usepackage{mathspec}
  \else
    \usepackage{fontspec}
  \fi
  \defaultfontfeatures{Ligatures=TeX,Scale=MatchLowercase}
\fi
% use upquote if available, for straight quotes in verbatim environments
\IfFileExists{upquote.sty}{\usepackage{upquote}}{}
% use microtype if available
\IfFileExists{microtype.sty}{%
\usepackage{microtype}
\UseMicrotypeSet[protrusion]{basicmath} % disable protrusion for tt fonts
}{}
\usepackage[margin=2.5cm]{geometry}
\usepackage{hyperref}
\hypersetup{unicode=true,
            pdftitle={Untitled},
            pdfauthor={Bruna Arbo Meneses},
            pdfborder={0 0 0},
            breaklinks=true}
\urlstyle{same}  % don't use monospace font for urls
\usepackage{graphicx,grffile}
\makeatletter
\def\maxwidth{\ifdim\Gin@nat@width>\linewidth\linewidth\else\Gin@nat@width\fi}
\def\maxheight{\ifdim\Gin@nat@height>\textheight\textheight\else\Gin@nat@height\fi}
\makeatother
% Scale images if necessary, so that they will not overflow the page
% margins by default, and it is still possible to overwrite the defaults
% using explicit options in \includegraphics[width, height, ...]{}
\setkeys{Gin}{width=\maxwidth,height=\maxheight,keepaspectratio}
\IfFileExists{parskip.sty}{%
\usepackage{parskip}
}{% else
\setlength{\parindent}{0pt}
\setlength{\parskip}{6pt plus 2pt minus 1pt}
}
\setlength{\emergencystretch}{3em}  % prevent overfull lines
\providecommand{\tightlist}{%
  \setlength{\itemsep}{0pt}\setlength{\parskip}{0pt}}
\setcounter{secnumdepth}{0}
% Redefines (sub)paragraphs to behave more like sections
\ifx\paragraph\undefined\else
\let\oldparagraph\paragraph
\renewcommand{\paragraph}[1]{\oldparagraph{#1}\mbox{}}
\fi
\ifx\subparagraph\undefined\else
\let\oldsubparagraph\subparagraph
\renewcommand{\subparagraph}[1]{\oldsubparagraph{#1}\mbox{}}
\fi

%%% Use protect on footnotes to avoid problems with footnotes in titles
\let\rmarkdownfootnote\footnote%
\def\footnote{\protect\rmarkdownfootnote}

%%% Change title format to be more compact
\usepackage{titling}

% Create subtitle command for use in maketitle
\providecommand{\subtitle}[1]{
  \posttitle{
    \begin{center}\large#1\end{center}
    }
}

\setlength{\droptitle}{-2em}

  \title{Untitled}
    \pretitle{\vspace{\droptitle}\centering\huge}
  \posttitle{\par}
    \author{Bruna Arbo Meneses}
    \preauthor{\centering\large\emph}
  \postauthor{\par}
      \predate{\centering\large\emph}
  \postdate{\par}
    \date{September 4, 2019}

\setlength\parindent{24pt}\setlength{\parskip}{0.0pt plus 1.0pt}

\begin{document}
\maketitle

\subsection{\texorpdfstring{\textsc{Resumo}}{Resumo}}\label{resumo}

Classificações de bacias hidrográficas baseadas em indicadores de
pressão ambiental permitem caracterizar gradientes de degradação
potencial que podem estar relacionados à conservação da biota. Por
exemplo, é plausível assumir que existe uma relação negativa entre
diversidade de espécies e pressão ambiental como fundamento para uso de
mapas de pressão na priorização de bacias para conservação, mas esse é
um pressuposto que deve ser validado, sob pena de o objetivo de
conservação (como diversidade) não ser efetivamente afetado por ações de
conservação baseadas nos fatores de pressão. Neste trabalho,
apresentamos um diagnóstico de pressão ambiental em 3359 bacias de 3ª
ordem no bioma Pampa e utilizamos estes resultados para testar a
existência de relação com características da ictiofauna. Definimos o
grau de pressão sobre as bacias como um índice global composto por seis
indicadores e realizamos projeções considerando cinco diferentes
configurações de pesos para esses indicadores. Consideramos como bacias
de referência aquelas que atenderam a uma das seguintes condições: 1)
apresentarem valor zero para todos os fatores de pressão, ou 2) estarem
entre as 10\% menos pressionadas em cada subunidade regional. O dados de
ictiofauna foram obtidos em 52 bacias hidrográficas de 3a ordem, sendo
cada bacia representada com amostragem por pesca elétrica em 150 m de
riacho. Foram ainda discriminadas 13 subunidades regionais definidas
pela combinação de ecorregiões aquáticas e fisionomias campestres. Todas
as bacias mostraram a presença de pelo menos um indicador de pressão,
sendo que bacias de referência foram identificadas somente em duas
subunidades regionais. Há uma proporção elevada de bacias (acima de
50\%) com níveis intermediários a altos de pressão no bioma Pampa, sendo
tais proporções variáveis entre subunidades fisionômicas de campo e
subunidades ecorregionais aquáticas. Ao buscar relações entre os fatores
de pressão e características da ictiofauna, verificamos que a composição
taxonômica está relacionada com agricultura e espelhos d'água
(representados principalmente por açudes). Verificamos ainda que a
riqueza taxonômica está relacionada negativamente com agricultura.
Riqueza funcional, composição funcional e porcentagens de espécies raras
e comuns não apresentaram relação com o gradiente de pressão. Os
resultados mostram que apenas parte das características ecológicas da
ictiofauna possui relação direta com grau de pressão ambiental em bacias
hidrográficas, indicando que a validade do uso de fatores de pressão
como proxies depende de quais características de ictiofauna são
consideradas. Portanto, para que os resultados de estratégias regionais
de conservação de ictiofauna sejam eficientes, o mapeamento de fatores
de pressão deve ser baseado na definição adequada de alvos e
complementado por testes sobre sua relação com os alvos de conservação.

\textsc{\textbf{Palavras-chave:}} planejamento sistemático de
conservação, campos sulinos, biodiversidade aquática, análise
multicritério, peixes.

\subsection{\texorpdfstring{\textsc{Abstract}}{Abstract}}\label{abstract}

Watershed classification using spatial representations of biodiversity
threatening factors are useful as proxies of disturbance gradients.
However, this is an assumption that should be tested in order to
effectively link conservation objectives to biodiversity outcomes. In
this study, we test whether stream fish community attributes are related
to environmental pressure in small watersheds of the Brazilian Pampa
grassland biome. Firstly, we estimated and mapped environmental pressure
in 3359 third-order watersheds, then we tested whether environmental
pressure predicted fish community attributes in 150-m stream segments
from 52 watersheds representing the gradient of environmental pressure.
Environmental pressure was defined by an index based on six indicators
(cropland, mining, urbanization, road density, dams and cattle density).
We defined reference watersheds as those where all the indicators were
equal to zero or which were ranked among the 10\% less disturbed in each
subregional Pampa unit (aquatic ecorregions and regional grassland
types). All watersheds had the influence of at least one disturbance
factor, and reference watersheds were indentified in only one
subregional unit. An elevated fraction of the watersheds (over 50\%)
showed intermediate to high pressure. Redundancy Analysis (RDA)
indicated that that taxonomic composition was related to cropland area
and dams. GLM results indicated that taxonomic richness is negatively
related to cropland area. Functional richness and composition, and
proportion of rare and common species were not related to the
environmental pressure gradient. Our results show that response of fish
communities to environmental pressure gradient may vary from positive,
to negative or no response, depending on the selected attribute.
Concerning the use in scientifically defensible conservation strategies,
mapping pressure indicators could be enhanced by testing their
relationship with conservation biodiversity outcomes.

\textsc{\textbf{Key words:}} systematic conservation planning, campos
sulinos, freshwater diversity, multicriteria analysis, fishes.

\subsection{Introdução}\label{introducao}

Os usos antrópicos dos recursos ambientais têm moldado em escala global
os padrões espaciais dos ecossistemas, determinando a magnitude da perda
de biodiversidade e de serviços ecossistêmicos e tornando raros os
locais no planeta que não sofrem algum grau de influência humana
(Sanderson et al. 2002; Vörösmarty et al. 2010). Para a maioria dos
ecossistemas, as atividades humanas que se caracterizam como as
principais ameaças a curto prazo são a expansão da agricultura e das
áreas urbanas (J. David Allan 2004; FAO 2010). A expansão da malha
rodoviária é também um fator de ameaça, seja pela implantação das
rodovias em si, seja pelo acesso que elas proporcionam a locais até
então inalterados (Trombulak and Frissell 2000; Sanderson et al. 2002;
{\textbf{???}}; Laurance and Balmford 2013). No Brasil, a previsão é de
que, até 2024, haja um aumento no uso da terra correspondente a 20\%
sobre a área média durante os anos de 2012 a 2014, especialmente para
cultivos de oleaginosas, cereais, cana e algodão (Vinet and Zhedanov
2016). Podemos esperar, consequentemente, um efeito em cascata no
sentido do investimento em infraestrutura e matéria-prima para produção
e distribuição desta atividade.

Devido a essa dinâmica em escala regional de expansão da ação humana, a
aplicação dos princípios de planejamento sistemático para conservação
tem crescido, com o intuito de localizar áreas prioritárias de forma
organizada e objetiva, contemplando o maior número de aspectos bióticos
e abióticos que definam interesse para conservação, como por exemplo,
áreas hipoteticamente intocadas ou pouco degradadas (Margules and
Pressey 2000; Linke, Turak, and Nel 2011; Lourival et al. 2011), áreas
de biodiversidade elevada ou biogeograficamente representativas, áreas
importantes para serviços ecossistêmicos, ou, áreas fortemente
pressionadas e que reúnam alguma das características anteriores (Myers
et al. 2000; {\textbf{???}}). O planejamento sistemático de conservação
tem como característica principal as escolhas claras e explícitas no
processo do planejamento tanto no que diz respeito as características
dos componentes da biodiversidade a serem usados no processo de
planejamento, quanto nos objetivos de conservação, sempre que possível
traduzidos em metas quantitativas e operacionais (R. L. Pressey et al.
2007). O uso dos sistemas de informação geográfica (SIG) facilita a
visualização dessas alterações no espaço permitindo maior eficiência e
sistematização nas relações das alterações na cobertura do solo na
biodiversidade e suas consequências para o planejamento territorial
(Vivitskaia JD Tulloch et al. 2015). Avaliações em escala espacial
ampla, de unidades de paisagem - como bacias hidrográficas - são
necessárias para gerenciamento do território, tomada de decisão e manejo
da fauna porque é a escala em que os efeitos cumulativos dos impactos
ficam evidentes, tornando as causas da degradação ambiental mais
facilmente observáveis (O'Neill et al. 1997; WIENS 2002; J. Stein,
Stein, and Nix 2002).

Quando o interesse de conservação está na biodiversidade aquática de
água-doce, as unidades espaciais empregadas na análise podem ser rios,
riachos ou bacias hidrográficas. Rios e riachos são geralmente afetados
por múltiplos distúrbios que se acumulam e interagem no espaço e no
tempo, tornando o entendimento da resposta para determinadas ameaças uma
tarefa complexa (J. David Allan 2004). É importante ressaltar que estes
indicadores de pressão representam uma degradação potencial do ambiente
e não da degradação efetiva, a qual é resultado da interação de fatores
cujo histórico, intensidade, frequência e efeito variam regionalmente.
De posse dessas informações, é possível determinar quais unidades de
paisagem recebem maior pressão por atividades antrópicas e assumir que
aquelas onde há ausência de pressão ambiental - ou pelo menos aquelas
menos pressionadas - sejam tratadas como sítios de referência para
comparações dos padrões observados ao longo de gradientes de pressão
(Stoddard et al. 2006). A condição destas unidades pode ser definida
como a similaridade do sítio avaliado em relação a um conjunto de sítios
menos degradados, levando em consideração características bióticas
(conjunto de indicadores, como o número de espécies intolerantes,
riqueza e composição de espécies; espécies ameaçadas, etc.) (Stoddard et
al. 2006; J. D. Allan et al. 2013).

Normalmente assume-se que existe uma relação negativa entre diversidade
de espécies e pressão ambiental como fundamento para uso de mapas de
pressão na priorização de áreas para conservação e que a proteção de
determinado conjunto de características abióticas (mais fáceis de
quantificar) consequentemente conservará o conjunto de espécies que
ocorram neste local (J. Stein, Stein, and Nix 2002; KHOURY, HIGGINS, and
WEITZELL 2011; Vivitskaia JD Tulloch et al. 2015). Entretanto, esse é um
pressuposto que deve ser validado para que o objetivo de conservação
seja efetivamente alcançado. Alguns trabalhos não consideram dados
bióticos na determinação de locais que sirvam como referência para
conservação com menor influência antrópica (Sanderson et al. 2002), e
outros consideram os alvos de conservação ao dar pesos aos indicadores
relacionados ao efeito potencial que eles tem sobre as espécies (J.
Stein, Stein, and Nix 2002; Heiner et al. 2011), mas poucos são os que
validam esta relação com dados de campo para mostrar que ela
verdadeiramente existe (Esselman et al. 2011; Ligeiro et al. 2013). A
comparação de dados de campo aumenta a eficiência das ações de
conservação ao esclarecer a relação do objetivo de conservação (por
exemplo, diversidade) e os fatores de pressão (Vivitskaia J. Tulloch et
al. 2013).

\subsubsection{A conservação de riachos nos campos do
Pampa}\label{a-conservacao-de-riachos-nos-campos-do-pampa}

Os ambientes campestres são extremamente sensíveis às mudanças da
cobertura do solo, não importa em que região do planeta estejam
localizados (Sala 2000; Andrade et al. 2015). Embora sejam ricos em
número de espécies (Gerhard Ernst Overbeck et al. 2005; G E Overbeck et
al. 2006; J. B. Wilson et al. 2012) e ofereçam inúmeros serviços
ecossistêmicos, com frequência são ignorados como alvo de conservação.
As regiões campestres localizadas na metade sul do Rio Grande do Sul
fazem parte do chamado Pampa, bioma de campos que inclui ainda partes do
Uruguai e Argentina (OVERBECK et al. 2007). A proporção de remanescentes
de campos é de aproximadamente 50\% em relação a sua área original no
Rio Grande do Sul, por conta da acelerada conversão do solo para,
principalmente, a agricultura e a silvicultura (Cordeiro and Hasenack
2009) e apenas 0,33\% dos campos estão atualmente protegidos em unidades
de proteção integral (OVERBECK et al. 2007).

Os principais usos presentes no bioma Pampa são pecuária e agricultura,
que podem aparecer em associação por rodízio de culturas (Nabinger,
Moraes, and Maraschin 2000). A agricultura apresentou crescimento mais
intenso nas últimos décadas, especialmente pelos cultivos de arroz,
milho, soja, trigo e silvicultura (IBGE, 2015) , atividades que contam
com incentivo de empresas privadas e do governo, com consequente
diminuição das áreas de vegetação nativa (Nabinger, Moraes, and
Maraschin 2000; OVERBECK et al. 2007; Cordeiro and Hasenack 2009). A
construção de açudes para dessedentação animal e para irrigação da
plantações é uma consequência dessas atividades econômicas, e
representam grave influência na fragmentação da rede hidrográfica
(Clavero and Hermoso 2011), uma vez que nem sempre necessitam de
licenciamento ambiental prévio para sua construção. A extração mineral
no RS tem no carvão o seu principal bem mineral, respondendo por 88\%
dos recursos de carvão do Brasil, para uso na geração termoelétrica e
metalúrgico (Rio Grande do Sul - SEPLAN 2013). Destaca-se também a
argila que ocorre junto as jazidas de carvão na região sudoeste para
fabricação de cerâmica e as rochas ornamentais (granitos e mármores),
cuja a produção concentra-se no centro-sul do Estado, especialmente para
extração de areia, saibro, argila e carvão para construção civil,
indústria e geração de energia.

\subsubsection{Degradação e conservação de ambientes
aquáticos}\label{degradacao-e-conservacao-de-ambientes-aquaticos}

Os ambientes aquáticos são especialmente impactados pelas alterações na
cobertura do solo em decorrência da concentração desproporcional da
ocupação humana nas proximidades dos recursos hídricos, onde as zonas
ripárias são extensamente modificadas mesmo em biomas espaçadamente
ocupados (Postel, Daily, and Ehrlich 1996; Sala 2000). Os rios são
partes funcionais da paisagem porque promovem conectividade pelas matas
ripárias, onde ocorrem trocas de materiais, de organismos, de energia e
recebem tudo que é escoado na área de captação da bacia (WIENS 2002). Em
comparação com os ambientes terrestres, os aquáticos estão menos
disponíveis em área, correspondendo a apenas 0,8\% da superfície do
planeta (D. Dudgeon et al. 2006). Assim, seu uso disseminado torna mais
crítico o fato de as espécies aquáticas apresentarem uma taxa de
extinção maior do que as espécies terrestres (Ricciardi and Rasmussen
1999; Sala 2000; Jenkins 2003; Vié, Hilton-Taylor, and Stuart 2009). Os
riachos, cursos d'água de pequeno a médio porte, por sua vez, são
ambientes de extrema importância por comporem grande parte da rede
hidrográfica, por sua alta produtividade e heterogeneidade de ambientes,
promovendo diversidade de habitats em comparação com rios maiores, além
de servirem como hábitat para espécies de peixes locais e migratórias.
Os rios que recebem a drenagem de riachos cujas bacias são
proporcionalmente mais alteradas pelo uso antrópico sofrem maiores
influências pelo efeito cumulativo, devido ao fluxo unidirecional das
redes hidrográficas (J. David Allan 2004). Essas influências podem
afetar a qualidade da água e do sedimento e a estabilização dos canais
existentes, podendo exercer efeito positivo quando em baixas
concentrações. Não por acaso, os processos ecológicos relacionados às
espécies de peixes continentais tem em sua conservação um considerável
desafio, devido à ação e interação de diferentes fatores de estresse que
promovem respostas complexas via caminhos distintos.

Dala-Corte et al. (2016), ao investigar os mecanismos pelos quais a
ictiofauna sofre influência do uso da terra nos campos do sul do Brasil,
mostraram diferentes caminhos pelos quais a conversão de paisagem
campestre em paisagem agrícola pode causar alterações nas comunidades de
peixes. Os efeitos da agricultura foram maiores sobre a diversidade
funcional, que, diminui com o incremento da degradação ripária e seus
efeitos sobre o aporte de sedimentos ao fundo dos riachos. Por outro
lado, os autores observaram um aumento na riqueza taxonômica de
espécies, vinculado às modificações no ambiente terrestre em diferentes
escalas espaciais. A efetividade das ações de conservação depende,
portanto, de mudanças de atitude em relação a biodiversidade aquática,
além da aceitação da bacia hidrográfica como unidade espacial de
planejamento, especialmente em relação as demandas conflitantes como
conservação das espécies, integridade do ecossistema e fornecimento de
bens e serviços para as populações humanas (D. Dudgeon et al. 2006).

Os riachos localizados no Pampa são os locais em cujas margens está
situada grande parte da vegetação florestal existente nessa região, mas
encontram-se em uma situação vulnerável devido as grandes transformações
regionais da paisagem (OVERBECK et al. 2007; Cordeiro and Hasenack
2009). A conversão da cobertura vegetal para outros usos, incluindo ou
não mudança na configuração das margens de rios e riachos - perda da
vegetação e erosão -, afetam diretamente os sistemas aquáticos e a
ictiofauna (Falcone, Carlisle, and Weber 2010). Portanto, para
fundamentar estratégias de conservação da hábitats e biota aquática no
Pampa, é preciso identificar elementos e padrões que auxiliem no
planejamento e na tomada de decisão para a conservação.

Este trabalho tem por objetivo caracterizar o grau de pressão antrópica
sobre bacias hidrográficas de riachos no Pampa sul-brasileiro e avaliar
a relação entre a pressão antrópica nessas bacias e características da
ictiofauna. Para isso, realizamos primeiramente um diagnóstico de
pressão ambiental em 3359 bacias de 3ª ordem no bioma Pampa, destacando
aquelas com potencial como referência para conservação devido ao baixo
grau de pressão. Os resultados deste diagnóstico foram utilizados para
testar a existência de relação entre pressão ambiental com as
características da ictiofauna: composição e riquezas taxonômica e
funcional; e proporção de espécies raras e comuns. Para isso,
desenvolvemos um índice global de pressão ambiental baseado em seis
indicadores antrópicos com diferentes combinações de pesos visando
testar diferentes potencialidades do efeito de cada um na ictiofauna.
Esta abordagem permite testar se as classificações baseadas no estado de
alteração das bacias estão relacionadas com a ictiofauna.

\section{Métodos}\label{metodos}

Neste trabalho, realizamos um diagnóstico do estado de pressão ambiental
nas bacias hidrográficas de riachos de 3ª ordem situadas no bioma Pampa
({\textbf{???}}) com base em seis indicadores de pressão ambiental:
agricultura, área urbana, mineração, espelhos d'água maiores do que 20
ha, densidade viária e densidade de gado (para detalhamento dos
indicadores, ver mais abaixo). Posteriormente, investigamos a relação
entre o gradiente de pressão ambiental das bacias e as riquezas e
composições taxonômica e funcional de peixes, além da proporção de
espécies raras e de espécies comuns. As bacias de 3ª ordem foram
derivadas de um modelo digital de elevação hidrologicamente consistido
(MDT-HC), para as quais foram quantificadas a presença de cada indicador
de pressão ambiental. Um índice global foi elaborado para quantificar a
pressão ambiental global em cada bacia através de métricas derivadas em
sistema de informação geográfica (SIG).

\subsection{Área de estudo e estrutura
espacial}\label{area-de-estudo-e-estrutura-espacial}

As análises envolveram a área do bioma Pampa !!!IBGE, 2004!!!, que se
estende por 176.476 km\(^2\) no estado do Rio Grande do Sul, sul do
Brasil (Figura 1). A cobertura vegetal desta região, zona de transição
entre os climas temperado e tropical, corresponde a um mosaico de campos
nativos, arbustos e manchas florestais que, desconsiderando os limites
políticos, estendem-se pelo Uruguai e parte da Argentina (OVERBECK et
al. 2007). As áreas campestres e florestais no Rio Grande do Sul sem
qualquer intervenção antrópica são muito pequenas, porque mesmo as áreas
sob algum tipo de proteção, como as unidades de conservação, apresentam
algum tipo de pressão antrópica, especialmente por pecuária (Cordeiro
and Hasenack 2009). O Rio Grande do Sul possui o 6\(^o\) efetivo bovino
no país, com números estáveis nas últimas décadas (1980-2014), embora
este rebanho esteja em franco crescimento no Brasil. Suínos (3\(^o\)
maior efetivo no Brasil) e ovinos (estado com maior número de cabeças)
são os outros rebanhos mais comuns !!!!IBGE, 2015!!!!.

A expansão da agricultura, que cobre aproximadamente 40\% da área do
bioma Pampa, representada especialmente pelas lavouras temporárias de
grãos como soja, trigo e arroz, compreende uma das principais ameaças à
conservação dos campos pelas intensas conversões no uso do solo que
acarretam (Cordeiro and Hasenack 2009); !!!!!IBGE 2015!!!!!). A
silvicultura também constitui atividade em expansão no Rio Grande do
Sul, com a plantação de eucalipto, acácia-negra e pinus (Hasenack et
al., n.d.) !!!!!(Hasenack et al., dados não publicados) (tabela x).

Como consequência, resta 31,38\% de sua cobertura com características
naturais ou seminaturais, se considerarmos o uso pecuário sobre o campo
nativo como fisionomia seminatural (Cordeiro and Hasenack 2009),
distribuídos em remanescentes campestres bastante fragmentados
(Contreras Osorio 2014). As conversões anteriormente citadas e a malha
viária, responsável pelo transporte de grãos, madeiras e outros bens
produzidos no Rio Grande do Sul, são as principais responsáveis pela
fragmentação do habitat, e podem funcionar como barreiras para certos
organismos, aumentando o status de fragmentação dos remanescentes já
pressionados (Teixeira 2015).

\textbackslash{}begin\{table{]}{[}ht{]} \centering
 \textbackslash{}caption\{Fisionomias campestres localizadas no Bioma
Pampa (Hasenack et al., dados não publicados). Relevo suave corresponde
a declividades entre 3 e 8\% e relevo ondulado corresponde a
declividades entre 8 e 20\%.{]}
\textbackslash{}begin\{tabular{]}\{p\{3cm{]} p\{5cm{]} p\{3cm{]}
p\{2cm{]}{]} \hline
 Fisionomia campestre \& Características principais \& Área (km\(^2\))
\& \% Pampa \textbackslash{} \hline
 Campo Litorâneo \& Terras baixas e relevo plano do litoral; Altitudes
inferiores a 30 m. \& 35.298,34 \& 20,00 \textbackslash{} Campo Misto
com Andropogôneas e Compostas \& Altitudes entre 30 e 400 m com relevo
suave (declividades entre 3 e 8\%). Depressão sedimentar com extensão
predominante leste-oeste. \& 35.221,13 \& 19,96 \textbackslash{} Campo
Arbustivo \& Altitudes entre 30 e 400 m com relevo ondulado
(declividades entre 8 e 20\%). Mosaico com floresta e campo onde o campo
predomina. Contém áreas de solos rasos e áreas com solos profundos de
baixa fertilidade. \& 30.126,80 \& 17,07 \textbackslash{} Campo com
Barba-de-Bode \& Altitudes entre 30 e 1.000 m e relevo suave
(declividades entre 3 e 8\%). Solos predominantemente profundos e de
baixa fertilidade. \& 21.005,35 \& 11,90 \textbackslash{} Campo com
Espinilho \& Altitudes entre 30 e 400 m sobre relevo suave (declividades
entre 3 e 8\%). Predominam solos férteis imperfeitamente a mal drenados.
\& 13.927,30 \& 7,89 \textbackslash{} Campo de Solos Rasos \& Altitudes
entre 30 e 400 m com relevo suave (declividades entre 3 e 8\%) e solos
rasos. \& 13.710,59 \& 7,77 \textbackslash{} Campo Graminoso \&
Altitudes entre 30 e 400 m com relevo suave (declividades entre 3 e
8\%). \& 11.522,56 \& 6,53 \textbackslash{} Floresta Estacional \&
Altitudes entre 30 e 400 m com relevo ondulado (declividades entre 8 e
20\%). Mosaico com floresta e campo onde a floresta predomina. \&
8.752,28 \& 4,96 \textbackslash{} Campo com Areais \& Altitudes entre 30
e 400 m com relevo suave (declividades entre 3 e 8\%). Os solos são
predominantemente arenosos, profundos, bem drenados e com baixa
fertilidade. \& 4.674,69 \& 2,65 \textbackslash{} Campo Misto do
Cristalino Oriental \& Altitudes entre 30 e 400 m e relevo suave
(declividades entre 3 e 8\%). \& 1.164,01 \& 0,66 \textbackslash{}
\hline
 \textbackslash{}end\{tabular{]}
\textbackslash{}label\{tab:fisonomias{]} \textbackslash{}end\{table{]}

\subsection{Delimitação das bacias de 3ª ordem e rede de
drenagem}\label{delimitacao-das-bacias-de-3-ordem-e-rede-de-drenagem}

As unidades espaciais básicas deste trabalho foram as bacias
hidrográficas a montante dos rios de 3ª ordem, ou seja, a área terrestre
que drena água e sedimento para os riachos de 3ª ordem. Os limites
destas bacias foram obtidos a partir do modelo digital de elevação
hidrologicamente consistido (MDE-HC) gerado com os pontos cotados,
curvas de nível e rede hidrográfica disponíveis na Base Cartográfica
Vetorial Contínua do Rio Grande do Sul - escala 1:50.000 (Hasenack and
Weber 2010). Para as bacias pertencentes a bacias localizadas na
fronteira com outros países (por exemplo, bacia do rio Quaraí), o MDE
utilizado foi o SRTM (Farr et al. 2007), disponibilizado em
\url{http://earthexplorer.usgs.gov/}. Tanto o MDE-HC quanto os limites
das bacias e a rede de drenagem foram processados através de uma série
de etapas realizados em sistema de informação geográfica (SIG),
utilizando a extensão ArcHydro 2.0, disponível para o software ArcGIS
10.3 (ESRI 2014). Apenas as bacias localizadas em território brasileiro
foram consideradas. Devido às dificuldades para gerar informações de
topologia em áreas planas do terreno (NARDI et al. 2008), as bacias
localizadas na planície costeira (Campo Litorâneo; ver Figura 1Figura 1)
\%figura foram excluídas das análises.

As 3359 bacias resultantes desse processo (Figura X) tem áreas entre
3,15 e 159,79 km\(^2\) (média = 25,94 km\(^2\);
\$\pm\(18,72 km\)\textsuperscript{2\(). Somadas, correspondem a uma área de 87.126,19 km\)}2\$
(49,63\% da área do bioma Pampa), das quais 71,30\% das bacias (2395)
tem área inferior a 30 km\(^2\). Para facilitar a comparação da pressão
ambiental nas bacias entre as fisionomias campestres respeitando os
limites biogeográficos de distribuição das espécies de peixes no Rio
Grande do Sul (Stoddard 2004), cada bacia foi classificada quanto a
fisionomia campestre e ecorregião aquática (Abell et al. 2008) onde está
localizada (Figura X). Como algumas fisionomias campestres podem estar
parcialmente contidas em mais de uma ecorregião, cada combinação de
fisionomia campestre com ecorregião aquática foi chamada de subunidade
regional. As fisionomias campestres analisadas neste trabalho têm áreas
entre 1.164,01 e 35.221,13 km\(^2\) e representam os sistemas ecológicos
que foram delimitados por similaridades em altitude, declividade, solo,
vegetação e uso da terra (Tabela
\textbackslash{}ref\{table:fisonomias{]}) {[}Hasenack{]} (Hasenack et
al., dados não publicados). As ecorregiões aquáticas Baixo Uruguai e
Laguna dos Patos possuem 97.477,96 e 141.844,57 km\(^2\) respectivamente
e foram definidas pelas similaridades na distribuição e composição de
espécies de peixes de água doce, incorporando grandes padrões evolutivos
e ecológicos (Abell et al. 2008). Identificar as bacias de 3ª ordem
usando mais de uma classificação quanto à região em que estão melhora a
capacidade de determinar indicadores de pressão agregados no espaço,
seja por questões históricas de uso e ocupação da terra ou por restrição
de condições ambientais, e oferece melhor detalhamento para tomada de
decisão e comparação entre condições de habitat.

\%aqui vai a figura 2

\subsection{Desenvolvimento do índice global de pressão
ambiental}\label{desenvolvimento-do-indice-global-de-pressao-ambiental}

O índice de pressão ambiental produzido neste trabalho foi criado
baseado em três premissas: 1) cada uso humano pode ter um impacto
potencial diferente no habitat e nos organismos presentes, e a magnitude
do impacto é definida inicialmente, através de pesos; 2) o impacto
potencial do indicador de pressão não tem relação com a distância em
relação ao rio a que ele está localizado, ou seja, duas bacias em que
uma área agrícola encontra-se distante do riacho terá o mesmo valor que
uma bacia com área agrícola do mesmo tamanho adjacente ao riacho como
resultado no índice global de pressão ambiental; 3) o aumento do impacto
potencial do indicadores resulta em prejuízo para a biota, ou seja,
sempre que houver um aumento na pressão ambiental, há uma diminuição nos
valores das métricas de caracterização da comunidade utilizados. Assim,
o índice global de pressão ambiental calculado neste trabalho é a soma
da área/densidade dos indicadores listados a seguir ponderada pela
potencialidade de impacto na ictiofauna.

Cada indicador poder exercer um efeito potencial de pressão ambiental
diferente nos ambientes aquáticos (J. Stein, Stein, and Nix 2002, J.
David Allan (2004), Falcone, Carlisle, and Weber (2010)), porém não é
possível estabelecer uma hierarquia de importância bem definida entre
esses indicadores. Por essa razão, estabelecemos cinco configurações de
ponderação distintas para os indicadores, de forma a produzir cinco
cenários para diagnóstico da pressão ambiental nas bacias e que sirvam
como \emph{proxy} do distúrbio antropogênico.

\subsection{Variáveis indicadoras de pressão
ambiental}\label{variaveis-indicadoras-de-pressao-ambiental}

Definimos o grau de pressão sobre as bacias como um índice global
composto por seis variáveis que representam uso antrópico disponíveis
para toda a extensão da área analisada, que estivessem em escalas
compatíveis para comparação e consideradas potencialmente impactantes
para o ambiente aquático. Este conjunto de variáveis inclui agricultura,
área urbana, mineração, espelhos d'água maiores do que 20 ha, densidade
viária e densidade de gado (Tabela
\textbackslash{}ref\{tab:my\_label{]}). Todas as fontes de dados estão
disponíveis publicamente, exceto a densidade de gado, que foi fornecida
pelo Instituto Brasileiro de Geografia e Estatística (IBGE; Censo
Agropecuário, 2006) como números absolutos por setor censitário com pelo
menos 10 informantes.

Os dados de áreas agrícolas e urbanas foram obtidos a partir do
mapeamento da vegetação do Rio Grande do Sul realizados por Cordeiro and
Hasenack (2009). O mapeamento, cuja escala é de 1:50.000 e no qual
somente remanescentes com um eixo maior do que 250 metros foram
vetorizados, foi realizado a partir de imagens LANDSAT 5 TM e 7 ETM+ (30
m de resolução espacial), ano base 2002 e possui 32 classes de uso e
cobertura de solo. Destas, cinco estão relacionadas a atividade
agropecuária com potencial impacto nos ambientes aquáticos
(``Agricultura sequeiro'', ``Agricultura irrigada'', ``Alagado arroz'',
``Misto campo/mato - Originalmente mata. Pastagem com domínio de campo
nativo sobre área desmatada'' e ``Uso misto - cultivo em pequenas
parcelas''). Estas categorias foram agrupadas em uma nova classe
denominada ``Agricultura''. A rede viária disponível digitalmente na
base cartográfica vetorial contínua do Rio Grande do Sul - escala
1:50.000 (Hasenack and Weber 2010) foi digitalizada sobre as cartas do
exercito de 1960 e está dividida em pavimentadas e não pavimentadas de
acordo com as jurisdições responsáveis por sua construção e manutenção:
municipal, estadual e federal. Conta também com as categorias
caminho/trilhas e rede ferroviária. Embora os dados sejam oriundos de
mapas da década de 1960, a pavimentação e a duplicação das rodovias
foram as principais mudanças na rede viária, não a expansão da rede
(Teixeira 2015). Todas as classes foram consideradas como igualmente
impactantes.

\textbackslash{}begin\{table{]}{[}ht{]} \centering
 \textbackslash{}caption\{Dados utilizados para obtenção dos indicadores
de pressão ambiental em bacias de 3a ordem no bioma Pampa, Brasil.{]}

\begin{verbatim}
\begin{tabular]{p{3cm] p{3cm] p{1.5cm] p{1.5cm] p{2cm]]
    \hline
    Dado original & Ameaça & Escala & Ano base & Fonte \\
    \hline
    Áreas agrícolas Escoamento de pesticidas e sedimentos
    Remoção da mata ripária & Aumento da temperatura da água & 1:250.000 & 2002 & \cite{Cordeiro2009a] \\
    Áreas urbanas & Aumento de superfície impermeável Canalização Alteração do fluxo Poluentes (esgoto doméstico, hospitalar e industrial) & 1:250.000 & 2002 & \cite{Cordeiro2009a] \\ %transformar esse texto em lista
    Rede viária & Cruzamentos Pontes com culvets ou pontilhões que causam barramento Remoção da mata ripária Aumento da temperatura da água Sedimentos & 1:50.000 & 1976-1984 & \cite{Hasenack2010b] \\
    Mineração & Extração do fundo Poluição & Não se aplica. & 2015 & \cite{DNPM2015] \\
    Espelhos dágua > 20 ha & Alteração do fluxo Barramento & 1:50.000 & 2003-2006 & \cite{FunacaoCearencedeMeteorologiaeRecursosHidricos-FUNCEME2008a] \\
    Efetivo gado & Aumento de nutrientes Erosão das margens Sedimentação do fundo Homogeneização dos sedimento & Não se aplica. & 2006 & \cite{IBGE2006] \\
    \hline
\end{tabular]
\label{tab:fatores]
\end{verbatim}

\textbackslash{}end\{table{]}

A variável Mineração foi obtida com base nos dados de processos
minerários disponibilizados pelo Departamento Nacional de Produção
Mineral (DNPM 2015), que possuem caráter informativo quanto aos
processos de concessão para atividade mineradora cadastrados no órgão
regulador. As poligonais cadastradas estão divididas em fases de
licenciamento: concessão de lavra, licenciamento, lavra garimpeira,
registro de extração, requerimento de lavra, requerimento de lavra
garimpeira, requerimento de licenciamento, requerimento de registro de
extração, requerimento de pesquisa, autorização de pesquisa e
disponibilidade. Foram incluídas nas análises as fases de concessão de
lavra, licenciamento, lavra garimpeira e registro de extração. Os
limites dos polígonos podem não representar precisamente os limites da
extração propriamente dita por serem informados pelos próprios
requerentes, e a qualidade das informações depende da metodologia
técnica utilizada para o cadastro. Os limites representam a área onde o
minerador pode realizar sua atividade, não a área já instalada
necessariamente. Mesmo assim, as poligonais dos processos minerários nas
fases consideradas representam áreas em que existe a prática da
atividade de mineração ou áreas onde ela pode ser implantada em um
futuro próximo.

Os dados dos efetivos de rebanho bovino, suíno e ovino do Rio Grande do
Sul foram fornecidos pelo IBGE e contém no número de cabeças por setor
censitário. Foram fornecidas apenas as informações dos setores
censitários com 10 informantes ou mais. Os setores censitários possuem
áreas muito variáveis, pois constituem a unidade territorial de coleta
das operações censitárias, definido pelo IBGE, com limites físicos
identificados com base na divisão político-administrativa do Brasil e no
número de habitantes. Mesmo com essa variação, em geral tem área maior
do que as bacias de 3ª ordem. Por isso, fizemos uma estimativa da
densidade dos rebanhos nas bacias calculando o número de cabeças de
bovino, ovino e suíno proporcionalmente a área do setor censitário
dentro de cada bacia.

Espelhos d'água é o indicador que corresponde aos açudes e
reservatórios, utilizados para dessedentação animal ou geração de
energia, por exemplo, e foi obtido a partir do Mapeamento dos Espelhos
d'água do Brasil com área superficial a partir de 20 hectares (Funação
Cearence de Meteorologia e Recursos Hídricos - FUNCEME 2008).
Originalmente, o mapeamento, que utilizou imagens dos satélites LandSAT
7 e CBERS, entre os anos 2003 e 2006, estava dividido por tipo de
espelho d'água: artificial (reservatório) ou natural (lagos, lagoas,
outros). Utilizamos apenas os espelhos d'água classificados com
artificiais.

Cada um dos indicadores abordados pode influenciar os ambientes
aquáticos de diferentes formas (Tabela
\textbackslash{}ref\{tab:fatores{]}). A agricultura, por exemplo,
usualmente ocupa a maior fração em uma bacia hidrográfica, enquanto
áreas urbanas e outros usos ocupam uma área menor. Entretanto, estes
usos podem exercer uma influência desproporcional nos ambientes
aquáticos, estejam eles próximos ou distantes dos corpos d'água (J.
David Allan 2004). Os cinco cenários diferenciam-se em função dos pesos
atribuídos a cada indicador de pressão (Tabela 3Tabela 3). No cenário 1,
todos os indicadores de pressão receberam o mesmo peso (0,17). No
cenário 2, tomamos por referência os pesos utilizados em J. Stein,
Stein, and Nix (2002). Em cada cenário, o somatório dos pesos de cada
indicador foi sempre igual a 1. O cenário 3 representa a média dos pesos
utilizados em quatro estudos: Sanderson et al. (2002), Heiner et al.
(2011), TURAK et al. (2011), Zhang and Chen (2014). Os cenários 4 e 5
foram elaboradas com o objetivo de simular quais seriam os resultados se
uma única variável recebesse um peso maior (0,50), enquanto as outras
fossem ponderadas da mesma forma (0,10). A variável agricultura foi a
que recebeu maior peso no cenário 4 e a variável área urbana foi a
definida no cenário 5. Todos os pesos somam 1 dentro de cada cenário.
Por fim, os valores calculados dos índices foram padronizados para que
variassem entre 0 a 1, onde 0 representa a ausência de pressão ambiental
(conforme os indicadores aqui empregados) e 1 representa pressão
ambiental máxima.

\textbackslash{}begin\{table{]}{[}h!{]} \centering
 \textbackslash{}caption\{Cenários (configurações de pesos) utilizados
na ponderação dos indicadores de pressão ambiental sobre bacias de 3a
ordem no bioma Pampa.{]} \textbackslash{}begin\{tabular{]}\{c c c c c
c{]} \hline
 Uso \& Cenário 1 \& Cenário 2 \& Cenário 3 \& Cenário 4 \& Cenário 5
\textbackslash{} \hline
 Agricultura \& 0,17 \& 0,19 \& 0,24 \& 0,50 \& 0,10 \textbackslash{}
Área urbana \& 0,17 \& 0,25 \& 0,18 \& 0,10 \& 0,50 \textbackslash{}
Rede viária \& 0,17 \& 0,16 \& 0,15 \& 0,10 \& 0,10 \textbackslash{}
Mineração \& 0,17 \& 0,15 \& 0,13 \& 0,10 \& 0,10 \textbackslash{}
Espelhos d'água \& 0,17 \& 0,07 \& 0,18 \& 0,10 \& 0,10 \textbackslash{}
Gado \& 0,17 \& 0,19 \& 0,12 \& 0,10 \& 0,10 \textbackslash{} \hline 
\textbackslash{}end\{tabular{]} \textbackslash{}label\{tab:cenarpesos{]}
\textbackslash{}end\{table{]}

\subsection{Determinação das bacias de
referência}\label{determinacao-das-bacias-de-referencia}

Para identificar as bacias de referência, as bacias de 3ª ordem foram
ranqueadas em ordem crescente quanto ao valor de cada indicador de
pressão e ao valor do índice global. Em seguida, assumimos que as bacias
nas quais todos os fatores de pressão medidos tiveram valor zero, foram
classificadas como ``condição de distúrbio mínimo'' (MDC, abreviação
para Minimum Disturbance Condition; Stoddard et al. (2006)), sendo essas
bacias as que representam a condição de referência ideal, mais próxima
da integridade biótica. A condição mais realística, no entanto, é aquela
em que se usa como referência as bacias com a melhor condição possível,
isto é, aquelas que apresentam a menor presença de fatores de pressão
entre as bacias avaliadas. Estas bacias foram classificadas como
``condição de menor distúrbio'' (LDC, abreviação para Least disturbed
condition; Stoddard et al. (2006)), descritas por um gradiente em que as
bacias menos pressionadas se aproximam das bacias em condição de
distúrbio mínimo em relação as bacias mais pressionadas. Neste trabalho,
as bacias que estiveram entre as 10\% menos pressionadas segundo o
índice global de pressão ao mesmo tempo em que apresentaram os menores
valores para cada indicador separadamente em cada subunidade regional
foram consideradas bacias em condição de menor distúrbio (LDC). LDC e
MDC foram consideradas bacias de referência. As bacias com índice global
entre 0,4 e 0,6 foram consideradas com pressão intermediária e as bacias
com valores maiores do que 0,6 foram consideradas como as mais
pressionadas.

\subsection{Relação entre fatores de pressão e
ictiofauna}\label{relacao-entre-fatores-de-pressao-e-ictiofauna}

Para analisar se as características da ictiofauna apresentam relação com
o grau de pressão ambiental na bacia, utilizamos dados de ictiofauna
coletados em 52 riachos no bioma Pampa. Como indicadores de resposta da
ictiofauna ao grau de pressão, utilizamos a composição e a riqueza
taxonômica (riqueza rarefeita) e a riqueza funcional de espécies, além
da proporção de espécies raras e comuns presentes em cada bacia
amostrada. A riqueza taxonômica rarefeita foi utilizada porque o número
de indivíduos amostrados por trecho de riacho foi bastante variável (de
105 a 1212 indivíduos), mesmo que a área amostrada tenha sido
semelhante. As espécies raras foram definidas como as que ocorreram em
menos de 10\% dos sítios (5 trechos amostrados), enquanto as espécies
comuns foram aquelas que ocorreram em número de sítios igual ou superior
a 50\%.

A composição de peixes foi determinada através de coleta com pesca
elétrica (EFKO GmbH model FEG 1500) em 52 sítios de amostragem (cada
sítio representa um riacho distinto), distribuídos por diferentes
sistemas campestres e por um gradiente de antropização. Cada sítio foi
amostrado uma única vez, em um trecho de 150 m no sentido
jusante-montante, onde as extremidades foram bloqueadas com redes para
evitar a fuga dos peixes. As coletas ocorreram entre os meses de outubro
e abril, de 2013 a 2015. Todos os indivíduos coletados foram
anestesiados com óleo de cravo, fixados em formol 10\% e preservados em
álcool 70\% para posterior identificação em laboratório (Comissão de
Ética no Uso de Animais da Universidade Federal do Rio Grande do Sul.
CEUA-UFRGS; \#24433). Os trechos de rio onde foram realizadas as coletas
tinham entre 0,89 e 10,26 m de largura (média = 4,85 m \(\pm\) 1,93) e
entre 4 e 65,24 cm de profundidade média (média = 30,73 cm \(\pm\)
13,74).

A diversidade e a riqueza funcionais foram obtidas através de uma matriz
sítio de coleta versus atributo, que foi calculada multiplicando uma
matriz de espécies versus atributos por uma matriz de sítios versus
espécies. Um conjunto de 13 atributos morfológicos descrevendo a função
trófica, a ocupação espacial na coluna d'água e o uso do hábitat: índice
de compressão corporal, altura relativa, posição do olho, posição da
boca, comprimento da cabeça, comprimento do pedúnculo, compressão do
pedúnculo, posição da nadadeira peitoral, área da nadadeira peitoral,
área da nadadeira ventral, área da nadadeira dorsal, área da nadadeira
caudal e biomassa (Dala-Corte et al. 2016) (Tabela 4). Um valor médio
foi calculado para cada atributo de cada espécie, baseado nas medidas de
cinco indivíduos representando diferentes classes de tamanho de cada
espécie nas amostras, sempre que possível. A composição funcional foi
descrita através do valor médio dos atributos de todas as espécies
presentes na comunidade (Community Weighted Mean traits, CWM)
{[}Lavorel2007a{]}. O espaço funcional preenchido pelas espécies de cada
sítio foi quantificado pelo índice de riqueza funcional descrito por
Villéger, Mason, and Mouillot (2008). Essa medida corresponde ao volume
do mínimo polígono convexo que engloba todas as espécies em um espaço
com número dimensões igual ao número de atributos medidos. Estas
análises foram realizadas com o pacote FD (Laliberté and Legendre 2010)
no software R (R Core Team 2016). \%ver ref

\textbackslash{}begin\{table{]}{[}h!{]} \centering
 \textbackslash{}caption\{Treze atributos funcionais indicadores de
uso/ocupação do habitat e comportamento alimentar. Ver Figura 3 para
abreviações.{]} \textbackslash{}begin\{tabular{]}\{p\{3cm{]} p\{3cm{]}
p\{5cm{]}{]} \hline
 Atributo \& Equação \& Função \textbackslash{} \hline
 Compressão do corpo \& \(BW/BD\) \& Relacionado a mobilidade. Corpos
comprimidos geralmente não encontrados em ambientes lênticos (Watson \&
Balon 1984). \textbackslash{} Profundidade relativa do corpo \&
\(BD/SL\) \& Inversamente relacionadas a velocidade do fluxo e determina
a habilidade de executar movimento vertical na coluna d'água (Gatz
1979). \textbackslash{} Tamanho relativo da cabeça \& \(HL/SL\) \&
Relacionado ao tamanho da presa (Gatz 1979). \textbackslash{} Posição
relativa do olho \& \(EH/HD\) \& Indica preferência de habitat vertical
na coluna d'água (Gatz 1979). Varia entre 0 e 1. Valores altos indicam
olhos superiores. \textbackslash{} Posição relativa da boca \& \(MH/HD\)
\& Indica posição vertical na qual o peixe forrageia (Albouy et al.
2011). Varia entre 0 e 1. Valores altos indicam boca superior.
\textbackslash{} Comprimento relativo do pedúnculo \& \(CpL/SL\) \&
Pedúnculo caudal mais longo indica boa habilidade para o nado (Gatz
1979). \textbackslash{} Compressão do pedúnculo \& \(CpH/CpW\) \&
Pedúnculo caudal comprimidos indicam atividade de nado pouco
desenvolvido (Gatz 1979). \textbackslash{} Posição peitoral \&
\(PfP/BD\) \& Relacionado a mobilidade (Dumay et al. 2004). Valores
altos indicam nadadeiras peitorais localizadas mais próximas ao dorso em
relação a parte mais ventral. \textbackslash{} Área relativa da
nadadeira peitoral \& \(PfL*PfH/SL\) \& Valores altos são indicativos de
espécies com hábitos bentônicos, nas quais as nadadeiras peitorais são
utilizadas como ancoras para resistirem em águas rápidas (Watson \&
Balon 1984). \textbackslash{} Área relativa da nadadeira ventral \&
\(VfL*VfH/\) \& Nadadeiras ventrais grandes são utilizadas como apoio
para espécies bentônicas {[}Casatti2006{]} \textbackslash{} Área
relativa da nadadeira caudal \& \(VfL*VfH/\) \& em ingles??? Large
caudal fins are associated to high propulsion (Gatz 1979).
\textbackslash{} Área relativa da nadadeira dorsal \& \(DfL*DfH/SL\)
Peixes que preferem fluxos rápidos geralmente tem nadadeiras dorsais
pequenas (Casatti \& Castro 2006). \textbackslash{} Biomassa \& Weight
(g) \& Indica contribuição ao sistema via metabolismo (Albouy et al.
2011). \textbackslash{} \hline
 \textbackslash{}end\{tabular{]} \textbackslash{}label\{tab:atribfunc{]}
\textbackslash{}end\{table{]}

\%figura 3

\subsection{Análise dos dados}\label{analise-dos-dados}

A influência dos indicadores de pressão na composições taxonômica e
funcional foi determinada através de analises de redundância parcial
(pRDA), utilizando bacia hidrográfica (Camaquã, Ibicuí etc.) como
covariável para controlar a variação. Realizamos duas pRDA separadas,
uma para composição taxonômica e outra para a composição funcional. Para
estas análises, apenas as espécies que ocorreram em mais de 5 sítios
foram consideradas.

Modelos lineares generalizados (GLM) foram utilizados para avaliar a
relação dos indicadores de pressão e com as riquezas taxonômica e
funcional, a proporção de espécies comuns e a proporção de espécies
raras encontradas em cada sítio. Além disso, foram realizados
separadamente modelos em que cada cenário dos índices globais de pressão
fosse a variável explicativa. Os valores de riqueza taxonômica possuem
distribuição aproximadamente normal. Assim, para esta variável, modelos
lineares simples foram ajustados. A família de modelos de regressão Beta
foi utilizada para as análises de modelos com as demais variáveis
resposta, por serem representadas por valores contínuos intervalo
unitário padrão (0,1). Se a variável resposta assumir os valores
extremos (0,1), a seguinte transformação foi realizada:
\((y*(n-1)+0,5)/n\), onde \textbackslash{}textit\{n{]} é o número de
riachos amostrados {[}Cribari-Neto2010{]}. O pacote vegan
{[}Oksanen2016{]} foi utilizado para os modelos lineares simples e o
pacote betareg {[}Cribari-Neto2010{]} \%ta certa essa ref? ela foi
citada na frase anterior tambem foi utilizado para os modelos cujas
variáveis resposta variam entre 0 e 1.

\section{Resultados}\label{resultados}

\subsection{Pressão ambiental sobre bacias de 3ª ordem no bioma
Pampa}\label{pressao-ambiental-sobre-bacias-de-3-ordem-no-bioma-pampa}

As 3359 bacias de 3ª ordem contém pelo menos um indicador de pressão
ambiental, sendo essas pressões distribuídas de forma desigual no bioma
Pampa (Figura X). Bacias com agricultura estão mais concentradas em
partes do sudoeste onde o relevo é ondulado ou há solos rasos. As
densidades de áreas urbanas e rede viária aumentam no sentido
oeste-leste, ou seja, mais aglomeradas à medida que se aproximam da
região metropolitana de Porto Alegre e do litoral, mais populosos do que
o interior do estado. Espelhos d'água estão associados à pecuária, para
dessedentação animal, e às áreas agrícolas irrigadas, encontrando-se,
portando, mais concentradas no oeste do bioma. A mineração está
majoritariamente localizada na região centro-sul do estado. A proporção
de áreas agrícolas e a densidade de rede viária foram os fatores de
pressão em média predominantes no bioma Pampa (0,38 e 0,59
respectivamente; Tabela 5).

Aproximadamente 20\% das bacias (702) têm 70\% ou mais da sua área
ocupada por agricultura, e 10,39\% (349) das bacias têm densidade viária
maior do que 1 km/km\(^2\), enquanto e a porcentagem de área ocupada
pelas poligonais destinadas a mineração variou entre 0 e 99,48\% (média
= 0,006 66 \(\pm\) 0,05). A porcentagem de área ocupada por mineração
foi menor a que 10\% em 98,29\% das bacias (Tabela 5). Não há
colinearidade forte entre os indicadores de pressão ambiental, sendo
agricultura e densidade de gado os indicadores mais correlacionados (r =
-0,26; p \(<\) 0,01). Agricultura, densidade viária e de gado são os
indicadores predominantes em todas as subunidades regionais (Figura 5).
Campo de Barba-de-Bode e Floresta Estacional são as subunidades
regionais com maior proporção média de área agrícola nas bacias. No
Campo com Barba-de-Bode localizados no sistema da Laguna dos Patos,
apenas 25\% das bacias tem menos de 71\% da área coberta por
agricultura. Campo Misto com Andropogôneas e Compostas, Campo Arbustivo
e Floresta Estacional tem as maiores densidades de rede viária.

\%figura 4

\%figura 5

\textbackslash{}begin\{table{]}{[}h!{]} \centering
 \textbackslash{}caption\{Estatística descritiva das variáveis
indicadoras de pressão ambiental.{]}
\textbackslash{}begin\{tabular{]}\{c c c c c c c{]} \hline
 Fator \& Unidade \& Média \& Mediana \& Desvio Padrão \& Mínimo \&
Máximo \textbackslash{} \hline
 Agricultura \& \% área da bacia \& 38,10 \& 33,20 \& 30,84 \& 0,00 \&
100,00 \textbackslash{} Espelhos dágua \& \% área da bacia \& 1,06 \&
0,00 \& 2,36 \& 0,00 \& 31,57 \textbackslash{} Mineração \& \% área da
bacia \& 0,66 \& 0,00 \& 5,42 \& 0,00 \& 99,48 \textbackslash{} Áreas
urbanas \& \% área da bacia \& 0,31 \& 0,00 \& 2,80 \& 0,00 \& 83,32
\textbackslash{} Rede viária \& km/km\(^2\) \& 0,59 \& 0,54 \& 0,32 \&
0,00 \& 2,65 \textbackslash{} Densidade de gado \& Número de
cabeças/km\(^2\) \& 0,35 \& 0,27 \& 0,30 \& 0,00 \& 1,92
\textbackslash{} \hline
 \textbackslash{}end\{tabular{]} \textbackslash{}label\{tab:my\_label{]}
\textbackslash{}end\{table{]}

De forma geral, há uma porção elevada de bacias de 3a ordem com níveis
intermediários a altos de pressão no bioma Pampa (Figura 6Figura 6).
Pelo menos 38,02\% das bacias apresentam um índice global de pressão
maior que 0,4 (intermediário) em todas os cenários, exceto no cenário 5,
em que esse percentual corresponde a 98,99\% (Figura 7). A distribuição
da frequência das bacias ao longo do gradiente de pressão não foi muito
diferente entre os cenários, mas variou entre as subunidades regionais,
influenciando a identidade das bacias com índice global com valores
intermediários mais do que as bacias nos extremos do gradiente. A
escolha dos pesos afeta a interpretação a respeito da pressão ambiental
sobre as bacias nas subunidades regionais apenas nos cenários 4 e 5,
pois são determinadas basicamente pela presença de agricultura e áreas
urbanas respectivamente. A distribuição da frequência de bacias no
gradiente de pressão é bastante variável entre subunidades regionais
(Figura 8), podendo chegar a 85\% de bacias com nível intermediário a
alto de pressão no Campo com Barba de Bode/Uruguai e na Floresta
Estacional (Tabela 6). Figuram também entre as subunidades mais
pressionadas o Campo com Espinilho, Campo Graminoso/Patos e Campo Misto
com Andropogôneas e Compostas. Campo Arbustivo, Campo com Areais e
Campos com Solos Rasos são as fisionomias menos pressionadas.
Considerando que os cenários não tem variações significativas quanto ao
diagnóstico global, apenas o cenário 2 foi utilizado para a descrição
dos resultados deste ponto em diante, por ser o cenário com pesos
variados que não tem agricultura com a maior penalidade. As figuras e
histogramas referentes a todos os cenários estão no Apêndice.

\%figura 6 \%figura 7 \%figura 8

Somente quatro bacias foram selecionadas como bacias de referência
segundo a condição de menor distúrbio (LDC) em todos os cenários, três
localizadas na subunidade regional Campo Arbustivo/Patos, e uma bacia na
subunidade Floresta Estacional/Patos (Tabela 6).

\%tabelas 6 e 7

As três bacias de 3ª ordem mais pressionadas foram as mesmas nos
cenários 1, 2 e 3, e estão situadas nas subunidades regionais
CEsp-Uruguai, e CArb-Patos. Nessas bacias, a porcentagem de agricultura
foi superior a 79\% e a densidade viária é considerada alta, entre 0,89
e 1,45 km/km\(^2\) (Forman and Hersperger 1996, Ripley, Scrimgeour, and
Boyce (2005)). No cenário 4, as três bacias mais pressionadas são
compostas por um conjunto de valores altos em áreas agrícolas, gado,
mineração ou rede viária, e estão localizadas nas subunidades
CEsp-Uruguai e CMAC-Patos, enquanto no cenário 5 destacou as bacias
dominadas por área urbana e rede viária, localizadas nas mesmas
subunidades.

\subsection{Relações entre pressão ambiental e
peixes}\label{relacoes-entre-pressao-ambiental-e-peixes}

Foram amostradas 112 espécies pertencentes a 16 famílias, sendo as
famílias mais ricas Characidae, Loricariidae e Cichlidae, com 38, 21 e
14 espécies respectivamente (65,17\% do número total de espécies). O
número médio de espécies por riacho foi de 18,83 (max = 32, min = 6). As
cinco espécies mais frequentes foram \emph{Heptapterus mustelinus}
(96,15\% dos sítios), \emph{Bryconamericus iheringii} (88,46\%),
\emph{Characidium pterostictum} (73,08\%), \emph{Rineloricaria stellata}
(63,46\%) e \emph{Astyanax laticeps} (56,68\%). Sessenta e cinco
espécies foram consideradas raras (pouco frequentes, ocorrendo em 5
riachos ou menos; 57,52\%), enquanto 10 espécies foram consideradas
comuns (ocorreram em mais da metade dos riachos; 8,85\%). Cinquenta e
quatro espécies (47,79\%) foram coletadas exclusivamente na ecorregião
do baixo Uruguai, 17 espécies foram coletadas exclusivamente na
ecorregião da Laguna dos Patos (15,04\%) e 42 espécies foram coletadas
nas duas ecorregiões (37,17\%).

Os resultados da pRDA mostraram que a composição taxonômica está
relacionada principalmente com agricultura e espelhos d'água. A relação
entre composição taxonômica das espécies de peixes e indicadores de
pressão explicou 48,77\% da variação dos dados, sendo que a maior parte
dessa explicação (37,84\%) foi atribuída às bacias hidrográficas
utilizadas como covariáveis. O primeiro eixo de ordenação explicou
34,37\% da variação da composição taxonômica, e o segundo eixo explicou
20\% dessa variação, os quais estão mais correlacionados com espelhos
d'água (r = 0,66) e áreas agrícolas (r = -0,53) respectivamente (Figura
9; Tabela 9). Estes resultados foram considerados significativos de
acordo com o teste de permutação de Monte Carlo (p = 0,018; 999
permutações) e, portanto, permitem estabelecer correlações entre os
indicadores de pressão e a ictiofauna.

O primeiro eixo de ordenação (RDA1) apresenta um gradiente associado a
presença de espelhos d'água, enquanto o segundo eixo (RDA2) apresenta um
gradiente associado principalmente a agricultura e mineração, mostrando
uma relação inversa quanto a composição de espécies nos trechos
amostrados cujas bacias apresentaram estes indicadores.

Os resultados da pRDA para composição funcional não foram considerados
significativos pelo mesmo teste de permutação mencionado anteriormente
(p = 0,958; 999 permutações), não permitindo uma interpretação confiável
sobre as relações entre as variáveis.

\%tabela 8 na dissetação \textbackslash{}begin\{table{]}{[}h!{]}
\centering
 \textbackslash{}begin\{tabular{]}\{c c c c{]} \hline
 Parametros \& RDA1 \& RDA2 \& RDA3 \textbackslash{} \hline
 Autovalores \& 0,3287 \& 0,1913 \& 0,1684 \textbackslash{} Correlações
espécies-habitat \& 0,84 \& 0,80 \& 0,69 \textbackslash{} \% variação
explicada (cumulativa) \& 34,39 \& 54,40 \& 72,01\textbackslash{} \hline
 \textbackslash{}end\{tabular{]} \textbackslash{}caption\{Análise de
redundância (pRDA) para relação entre composição taxonômica de peixes e
indicadores de pressão.{]}
\textbackslash{}label\{tab:pRDA\_TaxPressao{]}
\textbackslash{}end\{table{]}

\%tabela 9 \%figura nove

Três indicadores de pressão apresentaram relação com a riqueza
taxonômica: áreas urbanas, espelhos d'água e agricultura (Tabela 10). O
modelo explicou 26,90\% da variação da variável resposta, mas as
estimativas para espelhos d'água e áreas urbanas foram inverossímeis
devido à baixa quantidade de bacias amostradas com presença desses
indicadores (sobreajuste). Assim, não consideramos estas relações
interpretáveis. A agricultura está relacionada negativamente com riqueza
taxonômica (Figura 10Figura 10).

Nenhum indicador de pressão ou cenário teve relação significativa com
riqueza funcional (Tabela 11). Apesar disso, há um padrão de diminuição
da riqueza funcional em relação ao índice global de pressão, isto é, a
variação nos valores de riqueza funcional em bacias com baixa pressão
ambiental é maior do que nas bacias com pressão elevada (Figura 11).

\textbackslash{}begin\{table{]}{[}h!{]} \%tabela 10 na dissertacao
\centering
 \textbackslash{}caption\{Coeficientes beta para o modelo linear
generalizado do efeito de indicadores de pressão sobre riqueza
taxonômica de peixes em riachos do Pampa. Cinco modelos diferentes foram
realizados utilizando os cenários como variável explicativa, mas seus
coeficientes beta estão descritos em uma única tabela para facilitar a
representação.{]} \textbackslash{}begin\{tabular{]}\{c c c
c\textbar{}\textbar{}c c c c {]} \hline
 Indicadores de pressão \& Beta \& Erro padrão \& p-value \& Cenários \&
Beta \& Erro padrão \& \textbackslash{}textit\{p-value{]}
\textbackslash{} \hline
 Intercepto \& 14,53 \& 0,55 \& \(<\) 0,01 \& \& \& \textbackslash{}
Agricultura \& \textbackslash{}textbf\{-3,86{]} \& 2,25 \&
\textbackslash{}textbf\{0,05{]} \& 1 \& 1,125 \& 4,154 \& 0,78
\textbackslash{} Espelhos d'água \& \textbackslash{}textbf\{169,48{]} \&
70,80 \& \textbackslash{}textbf\{0,02{]} \& 2 \& 1,23 \& 4,103 \& 0,76
\textbackslash{} Rede viária \& 0,17 \& 2,55 \& 0,95 \& 3 \& -0,41 \&
3,26 \& 0,90 \textbackslash{} Gado \& 1,46 \& 2,13 \& 0,50 \& 4 \& -0,91
\& 2,59 \& 0,73 \textbackslash{} Mineração \& -25,55 \& 644,49 \& 0,97
\& 5 \& 2,028 \& 8,33 \& 0,73 \textbackslash{} Áreas urbanas \&
\textbackslash{}textbf\{1261,45{]} \& 517,67 \& 0,02 \& \& \&
\textbackslash{} \hline  \textbackslash{}end\{tabular{]}
\textbackslash{}label\{tab:my\_label{]} \textbackslash{}end\{table{]}

\%figura 10

\textbackslash{}begin\{table{]}{[}h!{]} \%tabela 11 \centering
 \textbackslash{}begin\{tabular{]}\{c c c c\textbar{}\textbar{}c c c c
{]} \hline
 Indicadores de pressão \& Beta \& Erro padrão \&
\textbackslash{}textit\{p-value{]} \& Cenários \& Beta \& Erro padrão \&
\textbackslash{}textit\{p-value{]} \textbackslash{} Intercepto \& -2,26
\& 0,13 \& \(<\) 0,01 \& \& \& \textbackslash{} Agricultura \& -0,44 \&
0,46 \& 0,34 \& 1 \& -0,83 \& 0,82 \& 0,31 \textbackslash{} Espelhos
dágua \& 43,13 \& 11,39 \& 0,53 \& 2 \& -0,80 \& 0,82 \& 0,98
\textbackslash{} Rede viária \& -0,10 \& 0,51 \& 0,83 \& 3 \& -0,97 \&
0,81 \& 0,14 \textbackslash{} Gado \& 0,21 \& 0,42 \& 0,61 \& 4 \& -0,89
\& 0,58 \& 0,26 \textbackslash{} Mineração \& 45,92 \& 121,34 \& 0,70 \&
5 \& -1.67 \& 1,79 \& 0,31 \textbackslash{} Área Urbana \& 53,46 \&
48,94 \& 0,27 \& \& \& \textbackslash{} \hline
 \textbackslash{}end\{tabular{]} \textbackslash{}caption\{Coeficientes
beta para o modelo linear generalizado do efeito de indicadores de
pressão sobre riqueza funcional de peixes em riachos do Pampa. Cinco
modelos diferentes foram realizados utilizando os cenários como variável
explicativa, mas seus coeficientes beta estão descritos em uma única
tabela para facilitar a representação.{]}
\textbackslash{}label\{tab:my\_label{]} \textbackslash{}end\{table{]}

\%figura 11

Nenhum indicador de pressão ou cenário apresentou relação com a
proporção de espécies comuns (Tabela 12) e raras (Tabela 13).

\textbackslash{}begin\{table{]}{[}h!{]} \%tabela 12 \centering
 \textbackslash{}begin\{tabular{]}\{c c c c\textbar{}\textbar{}c c c c
{]} \hline
 Indicadores de pressão \& Beta \& Erro padrão \& p-value \& Cenários \&
Beta \& Erro padrão \& p-value \textbackslash{} \hline
 Intercepto \& -0,48 \& 0,07 \& \(<\) 0,01 \& \& \& \& \textbackslash{}
Agricultura \& 0,13 \& 0,30 \& 0,65 \& 1 \& 0,26 \& 0,55 \& 0,63
\textbackslash{} Espelhos d'água \& -18,13 \& 10,28 \& 0,12 \& 2 \& 0,23
\& 0,55 \& 0,67 \textbackslash{} Gado \& -0,30 \& 0,29 \& 0,30 \& 3 \&
0,31 \& 0,43 \& 0,49 \textbackslash{} Rede viária \& 0,41 \& 0,34 \&
0,23 \& 4 \& 0,27 \& 0,34 \& 0,42 \textbackslash{} Mineração \& -8,49 \&
87,54 \& 0,92 \& 5 \& 0,52 \& 1,12 \& 0,64 \textbackslash{} Áreas
urbanas \& -26,91 \& 41,35 \& 0,52 \& \& \& \& \textbackslash{} \hline
 \textbackslash{}end\{tabular{]} \textbackslash{}caption\{Coeficientes
beta para o modelo linear generalizado do efeito de indicadores de
pressão sobre porcentagem de espécies comuns de peixes em riachos do
Pampa. . Cinco modelos diferentes foram realizados utilizando os
cenários como variável explicativa, mas seus coeficientes beta estão
descritos em uma única tabela para facilitar a representação.{]}
\textbackslash{}label\{tab:my\_label{]} \textbackslash{}end\{table{]}

\textbackslash{}begin\{table{]}{[}h!{]} \%tabela 13 \centering
 \textbackslash{}begin\{tabular{]}\{c c c c\textbar{}\textbar{}c c c c
{]} \hline
 Indicadores de pressão \& Beta \& Erro padrão \&
\textbackslash{}textit\{p-value{]} \& Cenários \& Beta \& Erro padrão \&
\textbackslash{}textit\{p-value{]} \textbackslash{} \hline
 Intercepto \& -1,80 \& 0,12 \& \(<\) 0,01 \& \& \& \& \textbackslash{}
Agricultura \& 0,06 \& 0,44 \& 0,89 \& 1 \& -0,30 \& 0,82 \& 0,71
\textbackslash{} Espelhos d'água \& 39,06 \& 11,95 \& 0,20 \& 2 \& -0,29
\& 0,81 \& 0,72 \textbackslash{} Gado \& 0,17 \& 0,42 \& 0,68 \& 3 \&
-0,56 \& 0,66 \& 0,39 \textbackslash{} Rede viária \& -0,45 \& 0,51 \&
0,38 \& 4 \& -0,62 \& 0,52 \& 0,23 \textbackslash{} Mineração \& -218,27
\& 155,54 \& 0,16 \& 5 \& -0,70 \& 1,65 \& 0,67 \textbackslash{} Áreas
urbanas \& -27,20 \& 59,66 \& 0,64 \& \& \& \& \textbackslash{} \hline
 \textbackslash{}end\{tabular{]} \textbackslash{}caption\{Coeficientes
beta para o modelo linear generalizado do efeito de indicadores de
pressão sobre porcentagem de espécies raras de peixes em riachos do
Pampa. Cinco modelos diferentes foram realizados utilizando os cenários
como variável explicativa, mas seus coeficientes beta estão descritos em
uma única tabela para facilitar a representação.{]}
\textbackslash{}label\{tab:my\_label{]} \textbackslash{}end\{table{]}

\section{Discussão}\label{discussao}

O diagnóstico de pressão ambiental mostrou que o bioma Pampa contém uma
porção elevada das bacias hidrográficas de 3ª ordem com níveis
intermediários a altos de pressão: em torno de 38\% das bacias,
dependendo do cenário, sofre pressão ambiental relativamente intensa por
pelo menos um indicador. Mesmo com a variação dessa proporção entre as
subunidades regionais, a maioria das fisionomias campestres é composta
apenas por bacias que apresentam pressão ambiental relativamente intensa
por pelo menos um indicador, enfraquecendo a possibilidade de servirem
como referência para conservação ou restauração dos campos. Apenas duas
das 12 subunidades ainda abrigam bacias de referência, as quais
representam menos de 1\% do número de bacias (e da área de bacias de 3ª
ordem) em cada fisionomia campestre (Tabela 6).

A comparação entre as cinco configurações de pesos não resultou em
diagnósticos com diferenças notáveis quanto a condição de pressão das
fisionomias campestres (com exceção do cenário 5; Figura A6Figura A6),
implicando em baixa sensibilidade dos resultados à atribuição de pesos
dos fatores. Mudar a importância relativa de um indicador, entretanto,
pode mudar a identificação das bacias em condição intermediária,
especialmente aquelas que não apresentam um único indicador de pressão
predominante, fragilizando o processo de destinação do investimento para
conservação. Agricultura, espelhos d'água e densidade da rede viária
foram os direcionadores de pressão nos três sistemas ecológicos mais
pressionados: Campo com Espinilho, Floresta Estacional e Campo com
barba-de-bode, mas o gradiente de pressão ambiental como um todo foi
determinado principalmente pela presença da agricultura, cuja grande
magnitude fez com que as bacias menos pressionadas tendessem a ser
aquelas localizadas nas fisionomias com menor vocação para atividades
agrícolas.

A riqueza taxonômica foi mais fortemente relacionada com agricultura,
uma relação com resultados que, como neste trabalho, tipicamente indicam
efeitos negativos sobre a ictiofauna (N. E. Roth, Allan, and Erickson
1996, Sala (2000), J. David Allan (2004)), mas que também pode
apresentar efeitos positivos sobre a riqueza de peixes (J. S. Harding et
al. 1998). A remoção da cobertura vegetal ripária e o aumento da entrada
de nutrientes oriunda dos resíduos agrícolas e urbanos causam alterações
no ambiente, como o aparecimento de macrófitas (Burrell et al. 2014), e
cria condições para que espécies nativas tipicamente encontradas em rios
maiores e espécies mais tolerantes a qualidade, temperatura e
modificações na estrutura do habitat ocupem esses ambientes (Scott and
Helfman 2001, Dala-Corte et al. (2016)). O que parece ser um efeito
positivo de aumento da riqueza em primeira instância, pode resultar em
subsequente diminuição no número de espécies causado pelo
desaparecimento das espécies mais sensíveis a essas novas condições
!!!!(Daga et al. 2012)!!!!!. A agricultura e os espelhos d'água também
estiveram relacionados com a composição taxonômica, evidenciando que
espécies mais tolerantes, como \emph{Synbranchus marmoratus} e
\emph{Characidium zebra} estão positivamente relacionadas com a presença
de espelhos d'água, diferentemente de \emph{Characidim pterostictum},
geralmente encontrada em arroios e rios com correnteza, fundo rochoso e
águas claras (Buckup and Reis 1997), no extremo oposto deste gradiente.

Embora os outros indicadores de pressão não tenham sido relacionados com
as características da ictiofauna, o aumento da área urbanizada pode
promover diferenças na composição e estrutura das assembleias de peixe,
onde apenas as espécies tolerantes e não-nativas tendem a ocorrer (Daga
et al. 2012). Além da fragmentação promovida pelo barramento da água, os
espelhos d'água promovem um novo ambiente tipicamente lêntico (Clavero
and Hermoso 2011), que ocupa uma grande área (144.836,220 ha em 2407
reservatórios; Funação Cearence de Meteorologia e Recursos Hídricos -
FUNCEME (2008)). Clavero and Hermoso (2011) não encontraram diferença na
riqueza de espécies em riachos livres ou com reservatórios, mas os
riachos diferenciaram-se quanto à composição das espécies: os riachos
barrados abrigavam mais espécies invasoras e menos espécies nativas. Os
efeitos da agricultura e dos barramentos por reservatórios lentamente
eliminam a heterogeneidade natural dos riachos, agindo como reguladores
artificiais da vazão (Poff et al. 2007, Dala-Corte et al. (2016)).
Nestes locais predominam as espécies cujas características indicam
ocuparem e se alimentarem na coluna d'água, em hábitats mais tipicamente
com pouca correnteza (espécies com corpo e pedúnculo caudal comprimidos,
olhos grandes, posição da boca terminal a superior e cabeça grande).

A necessidade de dados que cobrissem toda a área do bioma Pampa foi um
limitante na determinação dos indicadores que seriam usados neste
trabalho devido a disponibilidade dos dados. Apesar disso, o conjunto de
indicadores utilizados neste trabalho contempla as principais atividades
causadoras de pressão nas bacias do bioma Pampa. Uma abordagem
alternativa poderia diferenciar pressões causadas por diferentes tipos
de culturas agrícolas, de produção mineral ou estradas com diferentes
fluxos de veículos, que, somados, comporiam um novo indicador mais
complexo (p.ex. J. Stein, Stein, and Nix (2002)). O uso de abordagens
não-lineares, de outros indicadores de pressão (nutrientes, como
nitrogênio e fósforo, porcentagem de conversão e presença do gado
especificamente na zona ripária), de outras variáveis resposta (grau de
tolerância das espécies Esselman et al. (2011)), ou avaliação
multi-escala (N. E. Roth, Allan, and Erickson 1996, Sály et al. (2009),
Dala-Corte et al. (2016)) podem ser pontos a serem considerados para
compreender e caracterizar melhor as relações entre as alterações
antropogênicas em bacias hidrográficas e seus efeitos sobre
biodiversidade, de forma melhor sustentar medidas de conservação ou
restauração.

A ausência de relação entre o índice global de pressão ambiental e os
descritores da comunidade de peixes, assim como entre a riqueza
funcional e os indicadores de pressão mostraram que nem sempre as
relações entre a biodiverisdade e os fatores de pressão são lineares e
diretas. Ainda assim, podemos inferir um padrão de diminuição da riqueza
funcional com o aumento da pressão em todos os cenários, determinado
especialmente pela agricultura e pela densidade de gado (Figura 11).
HERBST et al. (2012), utilizando diferentes abordagens para medir o
efeito do pastoreio, encontraram uma relação de diminuição da riqueza de
invertebrados aquáticos em riachos com o aumento nos níveis dessa
atividade. O efeito da agricultura sobre a riqueza de espécies de peixe
é variável, dependente de escala e pode ser dar por diferentes
mecanismos.

O fato de a composição taxonômica dos peixes estar mais relacionada com
as bacias onde estão localizadas do que com os indicadores de pressão
mostra que há a um efeito importante de fatores biogeográficos
históricos na formação do pool de espécies de diferentes bacias,
aparentemente maior do que efeitos associados aos fatores de pressão
ambiental. Logo, a complexidade geográfica do bioma Pampa é um fator
importante a ser considerado e não surpreende que possa determinar, por
si só, grande parte da variação da comunidade aquática, pois tal
heterogeneidade determina barreiras geográficas e filtros ambientais
não-antrópogênicos para a distribuição das espécies.

As variáveis fisiográficas e os usos antrópicos raramente tem efeitos
isolados sobre o habitat e sobre a biota, mas produzem efeitos
sinérgicos e cumulativos tanto no entorno dos ambientes aquáticos quanto
em seus interiores, podendo interagir entre si. A baixa variação
explicada pelos indicadores de pressão individuais, ainda que
estatisticamente significativos, indica baixa sensibilidade das
variáveis resposta deste estudo às influências antropogênicas em uma
área tão ampla. Existem diversos fatores que influenciam as
características das comunidades locais: histórico de distúrbios dos
locais, variações de escala espacial mais restrita entre locais (em
função da declividade, posição na rede de drenagem) (p.ex Dala-Corte
(2016)). O fator biogeográfico regional foi mais importante que os
fatores de pressão na relação com a composição taxonômica. Além disso,
há um descompasso entre a data das coletas e as informações de uso do
solo, que pode ser responsável por uma subestimativa da proporção de
agricultura e áreas urbanas nas bacias.

\subsection{Implicações para gestão: conservação e
recuperação}\label{implicacoes-para-gestao-conservacao-e-recuperacao}

O diagnóstico de pressão ambiental do bioma Pampa e a identificação das
bacias de referência mostram o quanto a influência humana está
disseminada em todas as fisionomias campestres. Isso nos indica que
esforços de conservação devem ser focados no gerenciamento das
atividades antrópicas, especialmente na desaceleração das conversões da
cobertura do solo para agricultura, espelhos d'água e áreas urbanas.
Embora não contemple uma lista completa das ameaças existentes para o
bioma Pampa e seus riachos, este trabalho oferece uma importante base
para entender como estas pressões se distribuem e para onde os esforços
para conservação e gerenciamento da biodiversidade podem ser
direcionados. A visualização das bacias segundo um gradiente de pressão
resultante deste trabalho dispõe uma forma intuitiva de descrever a
intensidade de pressão nas bacias, facilitando comparações entre
condições posicionadas em um continuum. Isolar os sítios das pressões
antrópicas é muito pouco factível e, assim, a melhoria das condições dos
ambientes aquáticos depende de melhores praticas de gestão e melhorias
na configuração e gerenciamento da paisagem. Isso demanda que haja
planejamento e gestão regionalizada por órgãos gestores do ambiente, e
que este oriente a gestão e o manejo ao nível de propriedades rurais e
de pequenas unidades hidrográficas, que são as escalas geográficas em
que a prática do manejo e da restauração podem ser efetivadas.

Para que as ações de gerenciamento e restauração sejam efetivas, deve-se
diagnosticar a origem e/ou causa das ameaças além de mapeá-las, o que
requer um entendimento mais profundo dos mecanismos através dos quais o
uso da terra impacta os ecossistemas aquáticos. Diversos estudos tem
sido publicados nesse sentido (J. David Allan 2004, Ripley, Scrimgeour,
and Boyce (2005), Daniel et al. (2015), Dala-Corte et al. (2016)), mas
pouco (ou superficialmente) são incorporados nos estudos para
determinação de bacias de referência e áreas prioritárias para
conservação.

As ações para conservação e a escolha de seus alvos dependem da
intensidade da influência humana nas bacias. Quando o grau de pressão
ambiental é elevado (Figura 12), as ações são limitadas pela proporção
de áreas alvo para conservação disponíveis. O planejamento pode abordar
restauração dos ecossistemas, conectividade dos fragmentos remanescentes
e reintrodução das espécies (ou controle populacional, quando o aumento
da população torna-se um problema originado pelo aumento do impacto;
p.~ex: espécies invasoras). Onde o grau de pressão é relativamente mais
baixo (Figura 13), nas bacias de referência, por exemplo, uma grande
variedade de ações e alvos de conservação pode ser possível, podendo
incluir a criação e o gerenciamento de áreas com uso humano limitado (ou
seja, áreas protegidas) ou conscientização junto as comunidades humanas
locais. Bacias hidrográficas com níveis intermediários de influência
humana (Figura 14) são passíveis de estratégias mistas de conservação e
restauração, que são mais eficientes quando planejadas em escala
regional ou de paisagem (Noss 1983). Nessas áreas, frequentemente um
fator predomina sobre os outros e, assim, as medidas de conservação
poderiam ser planejadas no sentido destes usos.

\%figuras 12 - 15

\subsection{Considerações finais}\label{consideracoes-finais}

Nosso resultados mostraram que considerar a pressão no bioma Pampa como
um todo mascara possíveis variações entre subunidades regionais que
podem estar mais pressionadas do que outras. A proporção de bacias
hidrográficas com grau de pressão intermediário a alto é mais elevado
dentro de algumas fisionomias campestres do que considerando todas as
bacias do bioma Pampa somadas, e a existência de relação entre fator de
pressão com a ictiofauna depende da variável resposta escolhida.
Mapeamentos regionais geralmente não seguem uma especificidade
geográfica e de precisão de detalhe que deve ser assumida em estudos
focados na conservação da biodiversidade, especialmente em estudos
envolvendo ambientes aquáticos, tradicionalmente negligenciados nas
políticas de gerenciamento que abordam muito mais as causas e
consequências no ambiente terrestre. A multiplicidade de fontes de
informação por iniciativas nacionais implicam em diferenças no método de
geração dos mapeamentos e contribuem para a incerteza dos resultados.
Portanto, observar e considerar tais incertezas durante a execução dos
estudos é extremamente importante, porque uma das formas de lidar com
essa dificuldade ao relacionar informações originadas de formas
diferentes, aumentando o realismo da avaliação, é representar todos os
indicadores possíveis do conjunto de ameaças presentes nos habitats
aquáticos. Iniciativas estaduais com objetivos mais específicos e
focados na necessidade e no planejamento regional poderiam ser
fomentadas, utilizando proxies quando a informação direta não estiver
disponível, porém em nosso estudo mostramos que nem sempre as relações
com características ecológicas são simples ou diretas. Mesmo informações
adquiridas na fonte primária (como o efetivo de gado, neste trabalho),
são obtidas com unidade espacial pouco útil para fins de gerenciamento
de bacias.

\section{\texorpdfstring{\texttt{\{r\ child\ =\ "99-refs.Rmd"\}\ \#}}{\{r child = "99-refs.Rmd"\} \#}}\label{r-child-99-refs.rmd}

\section*{\texorpdfstring{\texttt{\{r\ child\ =\ "999-apendices.Rmd"\}\ \#}}{\{r child = "999-apendices.Rmd"\} \#}}\label{r-child-999-apendices.rmd}
\addcontentsline{toc}{section}{\texttt{\{r\ child\ =\ "999-apendices.Rmd"\}\ \#}}

\hypertarget{refs}{}
\hypertarget{ref-Abell2008b}{}
Abell, Robin, Michele L. Thieme, Carmen Revenga, Mark Bryer, Maurice
Kottelat, Nina Bogutskaya, Brian Coad, et al. 2008. ``Freshwater
Ecoregions of the World: A New Map of Biogeographic Units for Freshwater
Biodiversity Conservation.'' \emph{Bioscience} 58 (5): 403--14.
doi:\href{https://doi.org/10.1641/B580507}{10.1641/B580507}.

\hypertarget{ref-Allan2013}{}
Allan, J. D., P. B. McIntyre, S D P Smith, B S Halpern, G L Boyer, A.
Buchsbaum, G. A. Burton, et al. 2013. ``Joint analysis of stressors and
ecosystem services to enhance restoration effectiveness.'' \emph{Proc.
Natl. Acad. Sci.} 110 (1): 372--77.
doi:\href{https://doi.org/10.1073/pnas.1213841110}{10.1073/pnas.1213841110}.

\hypertarget{ref-Allan2004a}{}
Allan, J. David. 2004. ``Landscapes and Riverscapes: The Influence of
Land Use on Stream Ecosystems.'' \emph{Annu. Rev. Ecol. Evol. Syst.} 35
(1): 257--84.
doi:\href{https://doi.org/10.1146/annurev.ecolsys.35.120202.110122}{10.1146/annurev.ecolsys.35.120202.110122}.

\hypertarget{ref-Andrade2015a}{}
Andrade, Bianca O., Christiane Koch, Ilsi I. Boldrini, Eduardo
Vélez-Martin, Heinrich Hasenack, Julia-Maria Hermann, Johannes Kollmann,
Valério D. Pillar, and Gerhard E. Overbeck. 2015. ``Grassland
degradation and restoration: a conceptual framework of stages and
thresholds illustrated by southern Brazilian grasslands.'' \emph{Nat.
Conserv.} 13 (2). Associação Brasileira de Ciência Ecológica e
Conservação: 95--104.
doi:\href{https://doi.org/10.1016/j.ncon.2015.08.002}{10.1016/j.ncon.2015.08.002}.

\hypertarget{ref-Buckup1997}{}
Buckup, Paulo A, and Roberto E Reis. 1997. ``Characidiin Genus
Characidium (Teleostei, Characiformes) in Southern Brazil, with
Description of Three New Species.'' \emph{Copeia} 1997 (3): 531.
doi:\href{https://doi.org/10.2307/1447557}{10.2307/1447557}.

\hypertarget{ref-Burrell2014}{}
Burrell, Teresa K., Jonathan M. O'Brien, S. Elizabeth Graham, Kevin S.
Simon, Jon S. Harding, and Angus R. McIntosh. 2014. ``Riparian shading
mitigates stream eutrophication in agricultural catchments.''
\emph{Freshw. Sci.} 33 (1): 73--84.
doi:\href{https://doi.org/10.1086/674180}{10.1086/674180}.

\hypertarget{ref-Clavero2011}{}
Clavero, Miguel, and Virgilio Hermoso. 2011. ``Reservoirs promote the
taxonomic homogenization of fish communities within river basins.''
\emph{Biodivers. Conserv.} 20 (1): 41--57.
doi:\href{https://doi.org/10.1007/s10531-010-9945-3}{10.1007/s10531-010-9945-3}.

\hypertarget{ref-ContrerasOsorio2014}{}
Contreras Osorio, Ricardo. 2014. ``The Effect of Landscape Configuration
and Dispersal Capacity on Habitat Availability for Birds: An Assessment
for Campos Grasslands in Southern Brazil.'' PhD thesis, Universidade
Federal do Rio Grande do Sul.

\hypertarget{ref-Cordeiro2009a}{}
Cordeiro, José Luís Passos, and Heinrich Hasenack. 2009. ``Cobertura
vegetal atual do Rio Grande do Sul.'' In \emph{Campos Sulinos Conserv. E
Uso Sustentável Da Biodiversidade}, edited by Valério de Patta Pillar,
Sandra Cristina Müller, Zélia Maria de Souza Castilhos, and Aino Victor
Ávila Jaques, 285--99. Brasilia: Ministério do Meio Ambiente.

\hypertarget{ref-Daga2012a}{}
Daga, Vanessa Salete, Éder André Gubiani, Almir Manoel Cunico, and
Gilmar Baumgartner. 2012. ``Effects of abiotic variables on the
distribution of fish assemblages in streams with different anthropogenic
activities in southern Brazil.'' \emph{Neotrop. Ichthyol.} 10 (3):
643--52.
doi:\href{https://doi.org/10.1590/S1679-62252012000300018}{10.1590/S1679-62252012000300018}.

\hypertarget{ref-Dala-Corte2016a}{}
Dala-Corte, Renato B. 2016. ``Efeitos de processos regionais e locais
sobre comunidades, populações e interações em peixes de riachos.''
Doutorado, Universidade Federal do Rio Grande do Sul.
\url{http://doi.wiley.com/10.2307/1936608}.

\hypertarget{ref-Dala-Corte2016}{}
Dala-Corte, Renato B., Xingli Giam, Julian D. Olden, Fernando G. Becker,
Taís de F. Guimarães, and Adriano S. Melo. 2016. ``Revealing the
pathways by which agricultural land-use affects stream fish communities
in South Brazilian grasslands.'' \emph{Freshw. Biol.} 61 (11): 1921--34.
doi:\href{https://doi.org/10.1111/fwb.12825}{10.1111/fwb.12825}.

\hypertarget{ref-Daniel2015}{}
Daniel, Wesley M., Dana M. Infante, Robert M. Hughes, Yin-Phan Tsang,
Peter C. Esselman, Daniel Wieferich, Kyle Herreman, Arthur R. Cooper,
Lizhu Wang, and William W. Taylor. 2015. ``Characterizing coal and
mineral mines as a regional source of stress to stream fish
assemblages.'' \emph{Ecol. Indic.} 50 (September 2015): 50--61.
doi:\href{https://doi.org/10.1016/j.ecolind.2014.10.018}{10.1016/j.ecolind.2014.10.018}.

\hypertarget{ref-DNPM2015}{}
DNPM. 2015. ``Sistema de Informações Geográficas da Mineração -
SIGMINE.'' \url{http://sigmine.dnpm.gov.br/webmap/}.

\hypertarget{ref-Dudgeon2006a}{}
Dudgeon, David, Angela H Arthington, Mark O Gessner, Zen-Ichiro
Kawabata, Duncan J Knowler, Christian Lévêque, Robert J Naiman, et al.
2006. ``Freshwater biodiversity: importance, threats, status and
conservation challenges.'' \emph{Biol. Rev.} 81 (02): 163.
doi:\href{https://doi.org/10.1017/S1464793105006950}{10.1017/S1464793105006950}.

\hypertarget{ref-ESRI2014}{}
ESRI. 2014. \emph{ArcGIS 10.3}. Environmental Systems Research
Institute, Inc. Environmental Systems Research Institute, Inc.

\hypertarget{ref-Esselman2011b}{}
Esselman, Peter C., Dana M. Infante, Lizhu Wang, Dayong Wu, Arthur R.
Cooper, and William W. Taylor. 2011. ``An Index of Cumulative
Disturbance to River Fish Habitats of the Conterminous United States
from Landscape Anthropogenic Activities.'' \emph{Ecol. Restor.} 29
(1-2): 133--51.
doi:\href{https://doi.org/10.3368/er.29.1-2.133}{10.3368/er.29.1-2.133}.

\hypertarget{ref-Falcone2010}{}
Falcone, James A., Daren M. Carlisle, and Lisa C. Weber. 2010.
``Quantifying human disturbance in watersheds: Variable selection and
performance of a GIS-based disturbance index for predicting the
biological condition of perennial streams.'' \emph{Ecol. Indic.} 10 (2):
264--73.
doi:\href{https://doi.org/10.1016/j.ecolind.2009.05.005}{10.1016/j.ecolind.2009.05.005}.

\hypertarget{ref-FAO2010}{}
FAO. 2010. ``Global Forest Resources Assessment 2010.'' \emph{America
(NY).} 147: 350 pp.
doi:\href{https://doi.org/https://doi.org/10.1787/agr_outlook-2016-en}{https://doi.org/10.1787/agr\_outlook-2016-en}.

\hypertarget{ref-Farr2007}{}
Farr, Tom G., Paul A. Rosen, Edward Caro, Robert Crippen, Riley Duren,
Scott Hensley, Michael Kobrick, et al. 2007. ``The Shuttle Radar
Topography Mission.'' \emph{Rev. Geophys.} 45 (2): RG2004.
doi:\href{https://doi.org/10.1029/2005RG000183}{10.1029/2005RG000183}.

\hypertarget{ref-Forman1996a}{}
Forman, R. T. T., and A. M. Hersperger. 1996. ``Road ecology and road
density in different landscapes, with international planning and
mitigation solutions.'' \emph{Transp. Wildl. Reducing Wildl. Mortal.
Improv. Wildl. Passageways Across Transp. Corridors}, 23.
\url{http://trid.trb.org/view.aspx?id=475846}.

\hypertarget{ref-FunacaoCearencedeMeteorologiaeRecursosHidricos-FUNCEME2008a}{}
Funação Cearence de Meteorologia e Recursos Hídricos - FUNCEME. 2008.
``Mapeamento dos espelhos d'água no Brasil.''
\url{http://www2.ana.gov.br/Paginas/servicos/cadastros/Barragens/MapeamentoEspelhosDagua.aspx}.

\hypertarget{ref-Harding1998}{}
Harding, J S, E F Benfield, P V Bolstad, G S Helfman, and E. B. D.
Jones. 1998. ``Stream biodiversity: The ghost of land use past.''
\emph{Proc. Natl. Acad. Sci.} 95 (25): 14843--7.
doi:\href{https://doi.org/10.1073/pnas.95.25.14843}{10.1073/pnas.95.25.14843}.

\hypertarget{ref-Hasenack2010b}{}
Hasenack, Heinrich, and Eliseu (Org) Weber. 2010. ``Base cartográfica
vetorial contínua do Rio Grande do Sul - escala 1:50.000.''
UFRGS-IB-Centro de Ecologia.

\hypertarget{ref-Hasenack}{}
Hasenack, Heinrich, Eliseu Weber, Ilsi I. Boldrini, and Rafael Trevisan.
n.d. ``Mapa de sistemas ecológicos do estado do Rio Grande do Sul.
Unpublished.''

\hypertarget{ref-Heiner2011a}{}
Heiner, Michael, Jonathan Higgins, Xinhai Li, and Barry Baker. 2011.
``Identifying freshwater conservation priorities in the Upper Yangtze
River Basin.'' \emph{Freshw. Biol.} 56 (1): 89--105.
doi:\href{https://doi.org/10.1111/j.1365-2427.2010.02466.x}{10.1111/j.1365-2427.2010.02466.x}.

\hypertarget{ref-Herbst2012}{}
HERBST, DAVID B., MICHAEL T. BOGAN, SANDRA K. ROLL, and HUGH D. SAFFORD.
2012. ``Effects of livestock exclusion on in-stream habitat and benthic
invertebrate assemblages in montane streams.'' \emph{Freshw. Biol.} 57
(1): 204--17.
doi:\href{https://doi.org/10.1111/j.1365-2427.2011.02706.x}{10.1111/j.1365-2427.2011.02706.x}.

\hypertarget{ref-Jenkins2003}{}
Jenkins, Martin. 2003. ``Prospects for Biodiversity.'' \emph{Science
(80-. ).} 302 (5648): 1175--7.
doi:\href{https://doi.org/10.1126/science.1088666}{10.1126/science.1088666}.

\hypertarget{ref-Khoury2011a}{}
KHOURY, MARY, JONATHAN HIGGINS, and ROY WEITZELL. 2011. ``A freshwater
conservation assessment of the Upper Mississippi River basin using a
coarse- and fine-filter approach.'' \emph{Freshw. Biol.} 56 (1):
162--79.
doi:\href{https://doi.org/10.1111/j.1365-2427.2010.02468.x}{10.1111/j.1365-2427.2010.02468.x}.

\hypertarget{ref-Laliberte2010a}{}
Laliberté, Etienne, and Pierre Legendre. 2010. ``A distance-based
framework for measuring functional diversity from multiple traits.''
\emph{Ecology} 91 (1): 299--305.
doi:\href{https://doi.org/10.1890/08-2244.1}{10.1890/08-2244.1}.

\hypertarget{ref-Laurance2013}{}
Laurance, William F., and Andrew Balmford. 2013. ``A global map for road
building.'' \emph{Nature} 495 (7441): 308--9.
doi:\href{https://doi.org/10.1038/495308a}{10.1038/495308a}.

\hypertarget{ref-Ligeiro2013}{}
Ligeiro, Raphael, Robert M. Hughes, Philip R. Kaufmann, Diego R. Macedo,
Kele R. Firmiano, Wander R. Ferreira, Déborah Oliveira, Adriano S. Melo,
and Marcos Callisto. 2013. ``Defining quantitative stream disturbance
gradients and the additive role of habitat variation to explain
macroinvertebrate taxa richness.'' \emph{Ecol. Indic.} 25 (April 2016):
45--57.
doi:\href{https://doi.org/10.1016/j.ecolind.2012.09.004}{10.1016/j.ecolind.2012.09.004}.

\hypertarget{ref-Linke2011c}{}
Linke, Simon, Eren Turak, and Jeanne Nel. 2011. ``Freshwater
conservation planning: the case for systematic approaches.''
\emph{Freshw. Biol.} 56 (1): 6--20.
doi:\href{https://doi.org/10.1111/j.1365-2427.2010.02456.x}{10.1111/j.1365-2427.2010.02456.x}.

\hypertarget{ref-Lourival2011a}{}
Lourival, Reinaldo, Martin Drechsler, Matthew E. Watts, Edward T. Game,
and Hugh P. Possingham. 2011. ``Planning for reserve adequacy in dynamic
landscapes; maximizing future representation of vegetation communities
under flood disturbance in the Pantanal wetland.'' \emph{Divers.
Distrib.} 17 (2): 297--310.
doi:\href{https://doi.org/10.1111/j.1472-4642.2010.00722.x}{10.1111/j.1472-4642.2010.00722.x}.

\hypertarget{ref-Margules2000}{}
Margules, C R, and R. L. Pressey. 2000. ``Systematic conservation
planning.'' \emph{Nature} 405 (6783): 243--53.
doi:\href{https://doi.org/10.1038/35012251}{10.1038/35012251}.

\hypertarget{ref-Myers2000}{}
Myers, Norman, Russell A. Mittermeier, Cristina G Mittermeier, Gustavo
A. B. da Fonseca, and Jennifer Kent. 2000. ``Biodiversity hotspots for
conservation priorities.'' \emph{Nature} 403 (6772): 853--58.
doi:\href{https://doi.org/10.1038/35002501}{10.1038/35002501}.

\hypertarget{ref-Nabinger2000}{}
Nabinger, C, A. de Moraes, and G E Maraschin. 2000. ``Campos in Southern
Brazil.'' In \emph{Grassl. Ecophysiol. Grazing Ecol.}, edited by G.
Lemaire, J. Hodgson, A. de Moraes, C. Nabinger, and P. C. de F.
Carvalho, 1996:355--76. New York, NY: CAB International.

\hypertarget{ref-Nardi2008}{}
NARDI, FERNANDO, SALVATORE GRIMALDI, MONIA SANTINI, ANDREA PETROSELLI,
and LUCIO UBERTINI. 2008. ``Hydrogeomorphic properties of simulated
drainage patterns using digital elevation models: the flat area issue /
Propriétés hydro-géomorphologiques de réseaux de drainage simulés à
partir de modèles numériques de terrain: la question des zones planes.''
\emph{Hydrol. Sci. J.} 53 (6): 1176--93.
doi:\href{https://doi.org/10.1623/hysj.53.6.1176}{10.1623/hysj.53.6.1176}.

\hypertarget{ref-Noss1983}{}
Noss, Reed F. 1983. ``A Regional Landscape Approach to Maintain
Diversity.'' \emph{Bioscience} 33 (11): 700--706.
doi:\href{https://doi.org/10.2307/1309350}{10.2307/1309350}.

\hypertarget{ref-Overbeck2006}{}
Overbeck, G E, S C Müller, V D Pillar, and J Pfadenhauer. 2006.
``Floristic composition, environmental variation and species
distribution patterns in burned grassland in southern Brazil.''
\emph{Brazilian J. Biol.} 66 (4): 1073--90.
doi:\href{https://doi.org/10.1590/S1519-69842006000600015}{10.1590/S1519-69842006000600015}.

\hypertarget{ref-Overbeck2007a}{}
OVERBECK, G, S MULLER, A FIDELIS, J PFADENHAUER, V PILLAR, C BLANCO, I
BOLDRINI, R BOTH, and E FORNECK. 2007. ``Brazil's neglected biome: The
South Brazilian Campos.'' \emph{Perspect. Plant Ecol. Evol. Syst.} 9
(2): 101--16.
doi:\href{https://doi.org/10.1016/j.ppees.2007.07.005}{10.1016/j.ppees.2007.07.005}.

\hypertarget{ref-Overbeck2005}{}
Overbeck, Gerhard Ernst, Sandra Cristina Müller, Valério DePatta Pillar,
and Jörg Pfadenhauer. 2005. ``Fine‐scale post‐fire dynamics in southern
Brazilian subtropical grassland.'' \emph{J. Veg. Sci.} 16 (6): 655--64.
doi:\href{https://doi.org/10.1111/j.1654-1103.2005.tb02408.x}{10.1111/j.1654-1103.2005.tb02408.x}.

\hypertarget{ref-ONeill1997}{}
O'Neill, Robert V, Carolyn T Hunsaker, K Bruce Jones, Kurt H Riitters,
James D Wickham, Paul M Schwartz, Iris A Goodman, Barbara L Jackson, and
William S Baillargeon. 1997. ``Monitoring Environmental Quality at the
Landscape Scale.'' \emph{Bioscience} 47 (8): 513--19.
doi:\href{https://doi.org/10.2307/1313119}{10.2307/1313119}.

\hypertarget{ref-Poff2007}{}
Poff, N Leroy, Julian D Olden, David M Merritt, and David M Pepin. 2007.
``Homogenization of regional river dynamics by dams and global
biodiversity implications.'' \emph{Proc. Natl. Acad. Sci.} 104 (14):
5732--7.
doi:\href{https://doi.org/10.1073/pnas.0609812104}{10.1073/pnas.0609812104}.

\hypertarget{ref-Postel1996}{}
Postel, Sandra L., Gretchen C. Daily, and Paul R. Ehrlich. 1996. ``Human
Appropriation of Renewable Fresh Water.'' \emph{Science (80-. ).} 271
(5250): 785--88.
doi:\href{https://doi.org/10.1126/science.271.5250.785}{10.1126/science.271.5250.785}.

\hypertarget{ref-Pressey2007}{}
Pressey, Robert L., Mar Cabeza, Matthew E. Watts, Richard M. Cowling,
and Kerrie A. Wilson. 2007. ``Conservation planning in a changing
world.'' \emph{Trends Ecol. Evol.} 22 (11): 583--92.
doi:\href{https://doi.org/10.1016/j.tree.2007.10.001}{10.1016/j.tree.2007.10.001}.

\hypertarget{ref-Ricciardi1999a}{}
Ricciardi, Anthony, and Joseph B. Rasmussen. 1999. ``Extinction Rates of
North American Freshwater Fauna.'' \emph{Conserv. Biol.} 13 (5):
1220--2.
doi:\href{https://doi.org/10.1046/j.1523-1739.1999.98380.x}{10.1046/j.1523-1739.1999.98380.x}.

\hypertarget{ref-RioGrandedoSul-SEPLAN2013a}{}
Rio Grande do Sul - SEPLAN. 2013. ``Atlas Socioeconômico do Rio Grande
do Sul.'' \url{http://www.atlassocioeconomico.rs.gov.br}.

\hypertarget{ref-Ripley2005a}{}
Ripley, Travis, Garry Scrimgeour, and Mark S Boyce. 2005. ``Bull trout (
Salvelinus confluentus ) occurrence and abundance influenced by
cumulative industrial developments in a Canadian boreal forest
watershed.'' \emph{Can. J. Fish. Aquat. Sci.} 62 (11): 2431--42.
doi:\href{https://doi.org/10.1139/f05-150}{10.1139/f05-150}.

\hypertarget{ref-Roth1996}{}
Roth, Nancy E., J. David Allan, and Donna L. Erickson. 1996. ``Landscape
influences on stream biotic integrity assessed at multiple spatial
scales.'' \emph{Landsc. Ecol.} 11 (3): 141--56.
doi:\href{https://doi.org/10.1007/BF02447513}{10.1007/BF02447513}.

\hypertarget{ref-Sala2000a}{}
Sala, Osvaldo E. 2000. ``Global Biodiversity Scenarios for the Year
2100.'' \emph{Science (80-. ).} 287 (5459): 1770--4.
doi:\href{https://doi.org/10.1126/science.287.5459.1770}{10.1126/science.287.5459.1770}.

\hypertarget{ref-Sanderson2002}{}
Sanderson, Eric W., Malanding Jaiteh, Marc a. Levy, Kent H. Redford,
Antoinette V. Wannebo, and Gillian Woolmer. 2002. ``The Human Footprint
and the Last of the Wild.'' \emph{Bioscience} 52 (10): 891.
doi:\href{https://doi.org/10.1641/0006-3568(2002)052\%5B0891:thfatl\%5D2.0.co;2}{10.1641/0006-3568(2002)052{[}0891:thfatl{]}2.0.co;2}.

\hypertarget{ref-Saly2009}{}
Sály, Péter, Tibor Erős, Péter Takács, András Specziár, István Kiss, and
Péter Bíró. 2009. ``Assemblage level monitoring of stream fishes: The
relative efficiency of single-pass vs. double-pass electrofishing.''
\emph{Fish. Res.} 99 (3): 226--33.
doi:\href{https://doi.org/10.1016/j.fishres.2009.06.010}{10.1016/j.fishres.2009.06.010}.

\hypertarget{ref-Scott2001}{}
Scott, Mark C., and Gene S. Helfman. 2001. ``Native Invasions,
Homogenization, and the Mismeasure of Integrity of Fish Assemblages.''
\emph{Fisheries} 26 (11): 6--15.
doi:\href{https://doi.org/10.1577/1548-8446(2001)026\%3C0006:NIHATM\%3E2.0.CO;2}{10.1577/1548-8446(2001)026\textless{}0006:NIHATM\textgreater{}2.0.CO;2}.

\hypertarget{ref-Stein2002a}{}
Stein, J.L, J.A Stein, and H.A Nix. 2002. ``Spatial analysis of
anthropogenic river disturbance at regional and continental scales:
identifying the wild rivers of Australia.'' \emph{Landsc. Urban Plan.}
60 (1): 1--25.
doi:\href{https://doi.org/10.1016/S0169-2046(02)00048-8}{10.1016/S0169-2046(02)00048-8}.

\hypertarget{ref-Stoddard2005a}{}
Stoddard, John L. 2004. ``Use of Ecological Regions in Aquatic
Assessments of Ecological Condition.'' \emph{Environ. Manage.} 34 (S1):
S61--S70.
doi:\href{https://doi.org/10.1007/s00267-003-0193-0}{10.1007/s00267-003-0193-0}.

\hypertarget{ref-Stoddard2006}{}
Stoddard, John L., David P. Larsen, Charles P. Hawkins, Richard K.
Johnson, and Richard H. Norris. 2006. ``Setting expectations for the
ecological condition of streams: the concept of reference condition.''
\emph{Freshw. Bioassessment} 16 (August): 1267--76.
doi:\href{https://doi.org/https://doi.org/10.1890/1051-0761(2006)016\%5B1267:SEFTEC\%5D2.0.CO;2}{https://doi.org/10.1890/1051-0761(2006)016{[}1267:SEFTEC{]}2.0.CO;2}.

\hypertarget{ref-Teixeira2015}{}
Teixeira, Fernanda Zimmermann. 2015. ``Trilhando caminhos para avaliar
padrões espaciais de mortalidade e fragmentação em rodovias.''
PhD thesis, Universidade Federal do Rio Grande do Sul.
\url{https://www.lume.ufrgs.br/bitstream/handle/10183/131935/000979649.pdf?sequence=1}.

\hypertarget{ref-Trombulak2000}{}
Trombulak, Stephen C, and Christopher A Frissell. 2000. ``Review of
Ecological Effects of Roads on Terrestrial and Aquatic Communities.''
\emph{Conserv. Biol.} 14 (1): 18--30.
doi:\href{https://doi.org/10.1046/j.1523-1739.2000.99084.x}{10.1046/j.1523-1739.2000.99084.x}.

\hypertarget{ref-Tulloch2013}{}
Tulloch, Vivitskaia J., Hugh P Possingham, Stacy D Jupiter, Chris
Roelfsema, Ayesha I T Tulloch, and Carissa J Klein. 2013.
``Incorporating uncertainty associated with habitat data in marine
reserve design.'' \emph{Biol. Conserv.} 162 (June). Elsevier Ltd:
41--51.
doi:\href{https://doi.org/10.1016/j.biocon.2013.03.003}{10.1016/j.biocon.2013.03.003}.

\hypertarget{ref-Tulloch2015}{}
Tulloch, Vivitskaia JD, Ayesha IT Tulloch, Piero Visconti, Benjamin S.
Halpern, James EM Watson, Megan C. Evans, Nancy A. Auerbach, et al.
2015. ``Why do we map threats? Linking threat mapping with actions to
make better conservation decisions.'' \emph{Front. Ecol. Environ.} 13
(2): 91--99. doi:\href{https://doi.org/10.1890/140022}{10.1890/140022}.

\hypertarget{ref-Turak2011c}{}
TURAK, EREN, SIMON FERRIER, TOM BARRETT, EDWINA MESLEY, MICHAEL
DRIELSMA, GLENN MANION, GAVIN DOYLE, JANET STEIN, and GEOFF GORDON.
2011. ``Planning for the persistence of river biodiversity: exploring
alternative futures using process-based models.'' \emph{Freshw. Biol.}
56 (1): 39--56.
doi:\href{https://doi.org/10.1111/j.1365-2427.2009.02394.x}{10.1111/j.1365-2427.2009.02394.x}.

\hypertarget{ref-Vie2009}{}
Vié, Jean-Christophe, Craig Hilton-Taylor, and Simon N. Stuart. 2009.
\emph{Wildlife in a changing world}. Edited by Jean-Christophe Vié,
Craig Hilton-Taylor, and Simon N. Stuart. Vol. 51. 2. IUCN.
doi:\href{https://doi.org/10.2305/IUCN.CH.2009.17.en}{10.2305/IUCN.CH.2009.17.en}.

\hypertarget{ref-Villeger2008}{}
Villéger, Sébastien, Norman W H Mason, and David Mouillot. 2008. ``NEW
MULTIDIMENSIONAL FUNCTIONAL DIVERSITY INDICES FOR A MULTIFACETED
FRAMEWORK IN FUNCTIONAL ECOLOGY.'' \emph{Ecology} 89 (8): 2290--2301.
doi:\href{https://doi.org/10.1890/07-1206.1}{10.1890/07-1206.1}.

\hypertarget{ref-Vinet2016}{}
Vinet, Luc, and Alexei Zhedanov. 2016. \emph{OECD-FAO Agricultural
Outlook 2016-2025}. OECD-Fao Agricultural Outlook. Paris: OECD.
doi:\href{https://doi.org/10.1787/agr_outlook-2016-en}{10.1787/agr\_outlook-2016-en}.

\hypertarget{ref-Vorosmarty2010}{}
Vörösmarty, C J, P B McIntyre, M O Gessner, D Dudgeon, A Prusevich, P
Green, S Glidden, et al. 2010. ``Global threats to human water security
and river biodiversity.'' \emph{Nature} 467 (7315): 555--61.
doi:\href{https://doi.org/10.1038/nature09440}{10.1038/nature09440}.

\hypertarget{ref-Wiens2002}{}
WIENS, JOHN A. 2002. ``Riverine landscapes: taking landscape ecology
into the water.'' \emph{Freshw. Biol.} 47 (4): 501--15.
doi:\href{https://doi.org/10.1046/j.1365-2427.2002.00887.x}{10.1046/j.1365-2427.2002.00887.x}.

\hypertarget{ref-Wilson2012}{}
Wilson, J. Bastow, Robert K. Peet, Jürgen Dengler, and Meelis Pärtel.
2012. ``Plant species richness: the world records.'' Edited by Michael
Palmer. \emph{J. Veg. Sci.} 23 (4): 796--802.
doi:\href{https://doi.org/10.1111/j.1654-1103.2012.01400.x}{10.1111/j.1654-1103.2012.01400.x}.

\hypertarget{ref-Zhang2014}{}
Zhang, Haiping, and Liding Chen. 2014. ``Using the Ecological Risk Index
Based on Combined Watershed and Administrative Boundaries to Assess
Human Disturbances on River Ecosystems.'' \emph{Hum. Ecol. Risk Assess.
an Int. J.} 20 (6): 1590--1607.
doi:\href{https://doi.org/10.1080/10807039.2013.842746}{10.1080/10807039.2013.842746}.


\end{document}
