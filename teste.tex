\documentclass[]{article}
\usepackage{lmodern}
\usepackage{amssymb,amsmath}
\usepackage{ifxetex,ifluatex}
\usepackage{fixltx2e} % provides \textsubscript
\ifnum 0\ifxetex 1\fi\ifluatex 1\fi=0 % if pdftex
  \usepackage[T1]{fontenc}
  \usepackage[utf8]{inputenc}
\else % if luatex or xelatex
  \ifxetex
    \usepackage{mathspec}
  \else
    \usepackage{fontspec}
  \fi
  \defaultfontfeatures{Ligatures=TeX,Scale=MatchLowercase}
\fi
% use upquote if available, for straight quotes in verbatim environments
\IfFileExists{upquote.sty}{\usepackage{upquote}}{}
% use microtype if available
\IfFileExists{microtype.sty}{%
\usepackage{microtype}
\UseMicrotypeSet[protrusion]{basicmath} % disable protrusion for tt fonts
}{}
\usepackage[margin=1in]{geometry}
\usepackage{hyperref}
\hypersetup{unicode=true,
            pdftitle={Untitled},
            pdfauthor={Stéphane Laurent},
            pdfborder={0 0 0},
            breaklinks=true}
\urlstyle{same}  % don't use monospace font for urls
\usepackage{graphicx,grffile}
\makeatletter
\def\maxwidth{\ifdim\Gin@nat@width>\linewidth\linewidth\else\Gin@nat@width\fi}
\def\maxheight{\ifdim\Gin@nat@height>\textheight\textheight\else\Gin@nat@height\fi}
\makeatother
% Scale images if necessary, so that they will not overflow the page
% margins by default, and it is still possible to overwrite the defaults
% using explicit options in \includegraphics[width, height, ...]{}
\setkeys{Gin}{width=\maxwidth,height=\maxheight,keepaspectratio}
\IfFileExists{parskip.sty}{%
\usepackage{parskip}
}{% else
\setlength{\parindent}{0pt}
\setlength{\parskip}{6pt plus 2pt minus 1pt}
}
\setlength{\emergencystretch}{3em}  % prevent overfull lines
\providecommand{\tightlist}{%
  \setlength{\itemsep}{0pt}\setlength{\parskip}{0pt}}
\setcounter{secnumdepth}{0}
% Redefines (sub)paragraphs to behave more like sections
\ifx\paragraph\undefined\else
\let\oldparagraph\paragraph
\renewcommand{\paragraph}[1]{\oldparagraph{#1}\mbox{}}
\fi
\ifx\subparagraph\undefined\else
\let\oldsubparagraph\subparagraph
\renewcommand{\subparagraph}[1]{\oldsubparagraph{#1}\mbox{}}
\fi

%%% Use protect on footnotes to avoid problems with footnotes in titles
\let\rmarkdownfootnote\footnote%
\def\footnote{\protect\rmarkdownfootnote}

%%% Change title format to be more compact
\usepackage{titling}

% Create subtitle command for use in maketitle
\providecommand{\subtitle}[1]{
  \posttitle{
    \begin{center}\large#1\end{center}
    }
}

\setlength{\droptitle}{-2em}

  \title{Untitled}
    \pretitle{\vspace{\droptitle}\centering\huge}
  \posttitle{\par}
    \author{Stéphane Laurent}
    \preauthor{\centering\large\emph}
  \postauthor{\par}
      \predate{\centering\large\emph}
  \postdate{\par}
    \date{28 août 2018}

\usepackage{booktabs}
\usepackage{longtable}
\usepackage{array}
\usepackage{multirow}
\usepackage{wrapfig}
\usepackage{float}
\usepackage{colortbl}
\usepackage{pdflscape}
\usepackage{tabu}
\usepackage{threeparttable}
\usepackage{threeparttablex}
\usepackage[normalem]{ulem}
\usepackage{makecell}
\usepackage{xcolor}

\begin{document}
\maketitle

\hypertarget{table-including--symbol-escapef-not-set}{%
\section{Table including \%-symbol, escape=F not
set}\label{table-including--symbol-escapef-not-set}}

\begin{table}[!h]

\caption{\label{tab:unnamed-chunk-1}Caption}
\centering
\begin{threeparttable}
\begin{tabular}{lcccc}
\toprule
Variable & \textbackslash{}makecell[c]\{All\textbackslash{}\textbackslash{}(n = 300)\} & \textbackslash{}makecell[c]\{Group1\textbackslash{}\textbackslash{}(n = 120)\} & \textbackslash{}makecell[c]\{Group2\textbackslash{}\textbackslash{}(n = 100)\} & \textbackslash{}makecell[c]\{Group3\textbackslash{}\textbackslash{}(n = 80)\}\\
\midrule
Var1 & 31\% & 79\% & 51\% & 14\%\\
Var2 & 67\% & 42\% & 50\% & 43\%\\
Var3 & 14\% & 25\% & 90\% & 91\%\\
Var4 & 69\% & 91\% & 57\% & 92\%\\
\bottomrule
\end{tabular}
\begin{tablenotes}
\item \textit{Anmerkung:} 
\item * This is a note to show what * shows in this table plus some addidtional words to make this string a bit longer. Still a bit more
\end{tablenotes}
\end{threeparttable}
\end{table}

\hypertarget{table-including--symbol-escapef}{%
\section{Table including \%-symbol,
escape=F}\label{table-including--symbol-escapef}}

\begin{verbatim}
This leads to an Error, see pic below
\end{verbatim}

\begin{table}[!h]

\caption{\label{tab:unnamed-chunk-2}Caption}
\centering
\begin{threeparttable}
\begin{tabular}{lcccc}
\toprule
Variable & \makecell[c]{All\\(n = 300)} & \makecell[c]{Group1\\(n = 120)} & \makecell[c]{Group2\\(n = 100)} & \makecell[c]{Group3\\(n = 80)}\\
\midrule
Var1 & 31\% & 79\% & 51\% & 14\%\\
Var2 & 67\% & 42\% & 50\% & 43\%\\
Var3 & 14\% & 25\% & 90\% & 91\%\\
Var4 & 69\% & 91\% & 57\% & 92\%\\
\bottomrule
\end{tabular}
\begin{tablenotes}
\item \textit{Anmerkung:} 
\item * This is a note to show what * shows in this table plus some addidtional words to make this string a bit longer. Still a bit more
\end{tablenotes}
\end{threeparttable}
\end{table}


\end{document}
