\hypertarget{apendice}{%
\section*{Apêndice}\label{apendice}}
\addcontentsline{toc}{section}{Apêndice}

\begingroup\fontsize{10}{12}\selectfont

\begin{longtable}[t]{>{}c>{}c>{}c}
\caption{\label{tab:listaespecies}Lista das espécies de peixes coletadas nos 52 riachos no bioma Pampa, com indicação das famílias as
             quais pertencem e grupo definido conforme o numero de sítios onde ocorrem. Todas as espécies são nativas.}\\
\toprule
Família & Espécies & Grupo\\
\midrule
\endfirsthead
\caption[]{Lista das espécies de peixes coletadas nos 52 riachos no bioma Pampa, com indicação das famílias \textit{(continued)}}\\
\toprule
Família & Espécies & Grupo\\
\midrule
\endhead
\
\endfoot
\bottomrule
\endlastfoot
 & \textit{Bunocephalus doriae} & R\\

\multirow[t]{-2}{*}{\centering\arraybackslash Aspredinidae} & \textit{Pseudobunocephalus iheringii} & R\\
\cmidrule{1-3}
 & \textit{Aphyocharax anisitsi} & R\\

 & \textit{Astyanax fasciatus} & R\\

 & \textit{Astyanax saguazu} & R\\

 & \textit{Astyanax }sp. A & R\\

 & \textit{Astyanax }sp. B & R\\

 & \textit{Astyanax stenohalinus} & R\\

 & \textit{Cheirodon ibicuhiensis} & R\\

 & \textit{Cyanocharax tipiaia} & R\\

 & \textit{Diapoma speculiferum} & R\\

 & \textit{Heterocheirodon yatai} & R\\

 & \textit{Hyphessobrycon anisitsi} & R\\

 & \textit{Hyphessobrycon meridionalis} & R\\

 & \textit{Hyphessobrycon togoi} & R\\

 & \textit{Hypobrycon }sp. & R\\

 & \textit{Mimagoniates inequalis} & R\\

 & \textit{Moenkhausia dichroura} & R\\

 & \textit{Odontostilbe pequira} & R\\

 & \textit{Oligosarcus jacuiensis} & R\\

 & \textit{Oligosarcus oligolepis} & R\\

 & \textit{Oligosarcus robustus} & R\\

 & \textit{Oligosarcus }sp. & R\\

 & \textit{Serrapinnus calliurus} & R\\

 & \textit{Astyanax dissensus} & \\

 & \textit{Astyanax eigenmanniorum} & \\

 & \textit{Astyanax henseli} & \\

 & \textit{Astyanax jacuhiensis} & \\

 & \textit{Astyanax laticeps} & C\\

 & \textit{Astyanax procerus} & \\

 & \textit{Astyanax xiru} & \\

 & \textit{Bryconamericus iheringii} & C\\

 & \textit{Charax stenopterus} & \\

 & \textit{Cheirodon interruptus} & \\

 & \textit{Cyanocharax alegretensis} & \\

 & \textit{Cyanocharax uruguayensis} & \\

 & \textit{Diapoma terofali} & \\

 & \textit{Hyphessobrycon luetkenii} & \\

 & \textit{Oligosarcus jenynsii} & \\

\multirow[t]{-38}{*}{\centering\arraybackslash Characidae} & \textit{Pseudocorynopoma doriae } & C\\
\cmidrule{1-3}
 & \textit{Characidium tenue} & R\\

 & \textit{Characidium occidentale} & \\

 & \textit{Characidium orientale} & \\

 & \textit{Characidium pterostictum} & C\\

\multirow[t]{-5}{*}{\centering\arraybackslash Crenuchidae} & \textit{Characidium zebra} & \\
\cmidrule{1-3}
 & \textit{Cyphocharax saladensis} & R\\

 & \textit{Steindachnerina biornata} & R\\

 & \textit{Steindachnerina brevipinna} & R\\

 & \textit{Cyphocharax spilotus} & \\

\multirow[t]{-5}{*}{\centering\arraybackslash Curimatidae} & \textit{Cyphocharax voga} & \\
\cmidrule{1-3}
Erythrinidae & \textit{Hoplias malabaricus} & \\
\cmidrule{1-3}
 & \textit{Callichthys callichthys} & R\\

 & \textit{Corydoras }sp. & R\\

 & \textit{Corydoras undulatus} & R\\

\multirow[t]{-4}{*}{\centering\arraybackslash Callichthyidae} & \textit{Corydoras paleatus} & \\
\cmidrule{1-3}
 & \textit{Heptapterus }sp. & R\\

 & \textit{Heptapterus sympterygium} & R\\

 & \textit{Imparfinis mishky} & R\\

 & \textit{Rhamdella eriarcha} & R\\

 & \textit{Heptapterus mustelinus} & C\\

 & \textit{Pimelodella australis} & \\

 & \textit{Rhamdella longiuscula} & \\

\multirow[t]{-8}{*}{\centering\arraybackslash Heptapteridae} & \textit{Rhamdia quelen} & C\\
\cmidrule{1-3}
 & \textit{Hemiancistrus punctulatus} & R\\

 & \textit{Hisonotus armatus} & R\\

 & \textit{Hisonotus notopagos} & R\\

 & \textit{Hisonotus ringueleti} & R\\

 & \textit{Hypostomus aspilogaster} & R\\

 & \textit{Hypostomus uruguayensis } & R\\

 & \textit{Otocinclus arnoldi} & R\\

 & \textit{Otocinclus flexilis} & R\\

 & \textit{Pseudohemiodon laticeps} & R\\

 & \textit{Rineloricaria longicauda} & R\\

 & \textit{Rineloricaria }sp. & R\\

 & \textit{Rineloricaria strigilata} & R\\

 & \textit{Ancistrus brevipinnis} & \\

 & \textit{Ancistrus taunay} & C\\

 & \textit{Hemiancistrus fuliginosus} & \\

 & \textit{Hisonotus charrua} & \\

 & \textit{Hisonotus laevior} & \\

 & \textit{Hypostomus commersonii} & \\

 & \textit{Rineloricaria cadeae} & \\

 & \textit{Rineloricaria microlepidogaster} & \\

\multirow[t]{-21}{*}{\centering\arraybackslash Loricariidae} & \textit{Rineloricaria stellata} & C\\
\cmidrule{1-3}
Pseudopimelodidae & \textit{Microglanis cottoides} & \\
\cmidrule{1-3}
 & \textit{Homodiaetus anisitsi} & R\\

 & \textit{Ituglanis australis} & R\\

 & \textit{Ituglanis sp.} & R\\

 & \textit{Scleronema operculatum} & R\\

 & \textit{Scleronema }sp. & R\\

 & \textit{Scleronema aff. operculatum} & \\

\multirow[t]{-7}{*}{\centering\arraybackslash Trichomycteridae} & \textit{Scleronema minutum} & \\
\cmidrule{1-3}
Gymnotidae & \textit{Gymnotus carapo} & R\\
\cmidrule{1-3}
Hypopomidae & \textit{Brachyhypopomus bombilla} & R\\
\cmidrule{1-3}
Sternopygidae & \textit{Eigenmannia trilineata} & R\\
\cmidrule{1-3}
 & \textit{Cnesterodon decemmaculatus} & R\\

\multirow[t]{-2}{*}{\centering\arraybackslash Poeciliidae} & \textit{Phalloceros caudimaculatus} & \\
\cmidrule{1-3}
 & \textit{Apistogramma commbrae} & R\\

 & \textit{Australoheros scitulus} & R\\

 & \textit{Cichlasoma dimerus} & R\\

 & \textit{Crenicichla punctata} & R\\

 & \textit{Gymnogeophagus gymnogenys} & R\\

 & \textit{Gymnogeophagus labiatus} & R\\

 & \textit{Gymnogeophagus meridionalis} & R\\

 & \textit{Gymnogeophagus pseudolabiatus} & R\\

 & \textit{Australoheros facetus} & \\

 & \textit{Australoheros minuano} & \\

 & \textit{Crenicichla lepidota} & C\\

 & \textit{Crenicichla scotti} & \\

 & \textit{Gymnogeophagus mekinos} & \\

\multirow[t]{-14}{*}{\centering\arraybackslash Cichlidae} & \textit{Gymnogeophagus rhabdotus} & \\
\cmidrule{1-3}
Synbranchidae & \textit{Synbranchus marmoratus} & C\\*
\end{longtable}
\endgroup{}

\begin{figure}[H]

{\centering \includegraphics[width=0.7\linewidth]{figuras/A1_os5cenarios} 

}

\caption{Distribuição espacial do índice global de pressão ambiental resultante dos cinco cenários propostos.}\label{fig:distspaccen}
\end{figure}

\begin{figure}[H]

{\centering \includegraphics[width=1\linewidth]{figuras/A2_histogramascenario1} 

}

\caption[Frequência de bacias de 3$^a$ ordem ao longo do gradiente de pressão ambiental, de acordo com o cenário 1, em cada subunidade regional do bioma Pampa.]{Frequência de bacias de 3$^a$ ordem ao longo do gradiente de pressão ambiental, de acordo com o cenário 1, em cada subunidade regional do bioma Pampa. A linha cinza representa a porcentagem acumulada de bacias. CArb = Campo arbustivo, CAre = campo com areais, CBDB = Campo com Barba-de-bode, CEsp = Campo com espinilho, CSR = campo com solos rasos, CGra = Campo Graminoso, CMAC = Campo misto com Andropogôneas e Compostas, CMCO = Campo misto do cristalino oriental, FE = Floresta Estacional.}\label{fig:histogramascenario1}
\end{figure}

\begin{figure}[H]

{\centering \includegraphics[width=1\linewidth]{figuras/A3_histogramascenario2_2} 

}

\caption[Frequência de bacias de 3$^a$ ordem ao longo do gradiente de pressão ambiental, de acordo com o cenário 2, em cada subunidade regional do bioma Pampa.]{Frequência de bacias de 3$^a$ ordem ao longo do gradiente de pressão ambiental, de acordo com o cenário 2, em cada subunidade regional do bioma Pampa. A linha cinza representa a porcentagem acumulada de bacias. CArb = Campo arbustivo, CAre = campo com areais, CBDB = Campo com Barba-de-bode, CEsp = Campo com espinilho, CSR = campo com solos rasos, CGra = Campo Graminoso, CMAC = Campo misto com Andropogôneas e Compostas, CMCO = Campo misto do cristalino oriental, FE = Floresta Estacional.}\label{fig:histogramascenario2}
\end{figure}

\begin{figure}[H]

{\centering \includegraphics[width=1\linewidth]{figuras/A4_histogramascenario3} 

}

\caption[Frequência de bacias de 33$^a$ ordem ao longo do gradiente de pressão ambiental, de acordo com o cenário 3, em cada subunidade regional do bioma Pampa.]{Frequência de bacias de 3$^a$ ordem ao longo do gradiente de pressão ambiental, de acordo com o cenário 3, em cada subunidade regional do bioma Pampa. A linha cinza representa a porcentagem acumulada de bacias. CArb = Campo arbustivo, CAre = campo com areais, CBDB = Campo com Barba-de-bode, CEsp = Campo com espinilho, CSR = campo com solos rasos, CGra = Campo Graminoso, CMAC = Campo misto com Andropogôneas e Compostas, CMCO = Campo misto do cristalino oriental, FE = Floresta Estacional.}\label{fig:histogramascenario3}
\end{figure}

\begin{figure}[H]

{\centering \includegraphics[width=1\linewidth]{figuras/A5_histogramascenario4} 

}

\caption[Frequência de bacias de 3$^a$ ordem ao longo do gradiente de pressão ambiental, de acordo com o cenário 4, em cada subunidade regional do bioma Pampa.]{Frequência de bacias de 3$^a$ ordem ao longo do gradiente de pressão ambiental, de acordo com o cenário 4, em cada subunidade regional do bioma Pampa. A linha cinza representa a porcentagem acumulada de bacias. CArb = Campo arbustivo, CAre = campo com areais, CBDB = Campo com Barba-de-bode, CEsp = Campo com espinilho, CSR = campo com solos rasos, CGra = Campo Graminoso, CMAC = Campo misto com Andropogôneas e Compostas, CMCO = Campo misto do cristalino oriental, FE = Floresta Estacional.}\label{fig:histogramascenario4}
\end{figure}

\begin{figure}[H]

{\centering \includegraphics[width=1\linewidth]{figuras/A6_histogramascenario5} 

}

\caption[Frequência de bacias de 3$^a$ ordem ao longo do gradiente de pressão ambiental, de acordo com o cenário 5, em cada subunidade regional do bioma Pampa.]{Frequência de bacias de 3$^a$ ordem ao longo do gradiente de pressão ambiental, de acordo com o cenário 5, em cada subunidade regional do bioma Pampa. A linha cinza representa a porcentagem acumulada de bacias. CArb = Campo arbustivo, CAre = campo com areais, CBDB = Campo com Barba-de-bode, CEsp = Campo com espinilho, CSR = campo com solos rasos, CGra = Campo Graminoso, CMAC = Campo misto com Andropogôneas e Compostas, CMCO = Campo misto do cristalino oriental, FE = Floresta Estacional.}\label{fig:histogramascenario5}
\end{figure}


\end{document}
